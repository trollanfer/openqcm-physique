\q
L'eau est un solvant
\rv
polaire
\r
apolaire
\r
ionique
\q
Une molécule présentant des atomes d'électronégativité très différente peut être apolaire.
\rv
Vrai
\r
Faux
\q
Donner deux exemples de molécules apolaires non diatomiques
\q
Donner deux exemples de molécules polaires
\q
Dans une solution de chlorure de fer III (comprenant des ions Fe$^{3+}$ et Cl$^-$), il y a
\rv
trois fois plus d'ions chlorure que d'ions fer III
\r
trois fois plus d'ions fer III que d'ions chlorures
\r
autant des deux ions
\q
On dissout 1 mmol de $FeCl_3$ dans un litre d'eau pure. Alors
\rv
$C_{FeCl_3}$ = $10^{-3} \mpl$
\rv
$[Fe^{3+}] = 10^{-3} \mpl$
\rv
$[Cl^{-}] = 3.10^{-3} \mpl$
