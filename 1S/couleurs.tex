\q
Une banane éclairée en lumière 
\rv
jaune parait jaune
\rv
verte parait verte
\rv
rouge parait rouge
\r
bleue parait bleue
\q
Représenter les "cercles" de la synthèse additive
\q
Représenter les "cercles" de la synthèse soustractive
\q
On éclaire un écran bleu par deux spots rouge et vert. Il apparait
\r
rouge
\r
vert
\r
bleu
\r
blanc
\rv
noir
\r
autre réponse
\q
Citer les couleurs primaires de la synthèse additive
\q
La couleur des objets usuels (pull, table,...) s'explique par la synthèse additive.
\r
vrai
\rv
faux
\q
Une lumière blanche traverse un filtre absorbant le vert. Elle émerge
\rv
magenta
\r
jaune
\r
cyan
\r
autre couleur
\q
Il suffit de trois lasers monochromatiques pour restituer toutes les couleurs.
\rv
vrai
\r
faux
\q
A quoi correspond le daltonisme?
\q
Un drapeau français (bleu, blanc, rouge) est éclairé par un spot cyan. Il apparait
\r
(bleu, cyan, magenta)
\r
(noir, cyan, rouge)
\rv
autre réponse  
\q
La perception des couleurs par l'oeil repose sur la synthèse soustractive
\rv
faux
\r
vrai
\q
On superpose deux filtres magenta et vert, on observe du
\rv
noir
\r
vert
\r
magenta
\r
autre couleur
\q
Citer les couleurs primaires de la synthèse soustractive
\q
Une observation au microscope d'un écran fait apparaître les pixels rouge et vert allumés. L'écran est
\rv
jaune
\r
cyan
\r
magenta
\r
autre couleur
\q
On réalise la chromatographie sur couche mince d'une encre noire. Celle-ci peut ne faire apparaître que deux taches bleues et jaune.
\rv
vrai
\r
faux
\q
On réalise la chromatographie sur couche mince d'une encre noire. Celle-ci fait apparaître au moins deux taches colorées.
\r
vrai
\rv
faux
\q
Si l'on mélange sur la palette deux peintures complémentaires, on obtient du noir
\rv
Vrai
\r
Faux
\q
A la traversée d'une solution bleue comme le sulfate de cuivre, le spectre de la lumière incidente :
\r
présente une bande noire dans le bleu
\r
présente des raies noires dans le bleu
\r
présente des raies brillantes dans le bleu
\rv
présente une bande brillante dans le bleu.
\q
Un pixel d'ordinateur a ses composantes bleu et rouge actives. On observera ce pixel
\r
jaune
\r
cyan
\rv
magenta
\r
bleu
\r
rouge
\q
La couleur d'un objet ne dépend que de sa couleur
\rv
Faux
\r
Vrai
\q
Un filtre cyan
\rv
laisse passer le vert
\r
absorbe vert et bleu
\r
laisse passer le rouge
\q
Un écran blanc éclairé par trois faisceaux respectivement rouge, bleu, et vert parait
\rv
blanc
\r
noir
\q
Pour tracer un trait bleu, une imprimante utilise des pigments
\rv
magenta et cyan
\r
magenta et jaune
\r
jaune et cyan
\q
Un écran magenta éclairé par deux faisceaux respectivement rouge et bleu parait
\rv
magenta
\r
bleu
\r
rouge
\r
autre couleur
\q
Un écran magenta éclairé par deux faisceaux respectivement rouge et vert parait
\r
magenta
\r
jaune
\rv
rouge
\r
autre couleur
\q
De quelle couleur est un faisceau de lumière blanche, après la traversée d'un filtre absorbant le bleu?
\rv
jaune
\r
bleu
\r
autre couleur
\q
De quelle couleur est un faisceau de lumière blanche, après la traversée d'un filtre absorbant le cyan?
\rv
rouge
\r
cyan
\r
magenta
\r
jaune
\q
Une lumière blanche traverse deux filtres superposés magenta et vert. La lumière émergente est :
\rv
noire
\r
magenta
\r
verte
\q
Une lumière blanche traverse un filtre rouge. La lumière émergente est 
\rv
rouge
\r
cyan
\q
Le magenta a pour longueur d'onde :
\rv
n'a pas de longueur d'onde associée.
\r
la moyenne des longueurs d'onde du bleu et du rouge, soit 600nm.
\r
750nm




