%%%%%%%%%%%%%%%%Cohésion matiere
\q
Citer les quatres interactions fondamentales
\q
La portée de l'interaction forte est
\rv
limitée au noyau
\r
infinie
\q
La gravitation est totalement négligeable à l'échelle microscopique
\rv
Vrai
\r
Faux
\q
L'interaction forte prédomine à l'échelle moléculaire
\rv
Faux
\r
Vrai
\q
La cohésion du noyau est assurée par :
\rv
l'interaction forte
\r
l'interaction électrique
\r
l'interaction gravitationnelle
\q
Le nombre de nucléons de $_9^4X$ est 
\rv
9
\r
4
\r
5
\q 
Au niveau microscopique, l'interaction gravitationnelle est 
\rv
négligeable devant l'interaction électrique
\r
comparable à l'interaction électrique
\r
prépondérante devant l'interaction électrique
\q
Définir deux isotopes. Donner un exemple.
\q
On approche une sphère conductrice isolée, initialement neutre, d'un corps isolant chargé positivement fixe.
\r
rien ne se passe
\r
le métal est repoussé
\r
le métal est attiré et se colle
\rv
le métal est attiré, touche la sphère et est repoussé.
\q
On approche sans contact une sphère conductrice isolée, initialement neutre, d'un corps isolant chargé positivement fixe.
\r
Rien ne se passe
\r
des charges positives "apparaissent" en face du corps isolant
\rv 
des charges positives "apparaissent" à l'opposé du corps isolant
\q
On approche sans contact une sphère conductrice reliée à la terre, initialement neutre, d'un corps isolant chargé positivement fixe.
\r
rien ne se passe
\rv
la sphère se charge négativement
\r
la sphère se charge positivement
\q
Un corps chargé négativement a perdu des électrons
\rv
Faux
\r
Vrai
\q
Enoncer la loi de Coulomb
\q
Enoncer la loi de la gravitation universelle
\q
C'est parce que l'eau est conductrice qu'un filet d'eau peut être dévié par un corps chargé.
\rv
Faux
\r
Vrai
\q
La charge électrique totale d'un corps isolé est nécessairement constante
\rv
Vrai
\r
Faux
\q
Quelle est l'interaction responsable de la cohésion des objets qui nous entourent ?
\r
l'interaction forte
\rv
l'interaction électrique
\r
l'interaction gravitationnelle




