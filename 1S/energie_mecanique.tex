%%%%%%%%%%%%%%%% Energie
\q
Si un solide est en mouvement rectiligne uniforme
\r
son énergie mécanique est forcément constante
\r
son énergie potentielle est forcément constante
\rv
son énergie cinétique est forcément constante
\r
autre réponse
\q
Si l'énergie mécanique totale d'un cycliste se conserve lors d'une descente, sa vitesse augmente forcément.
\rv
Vrai
\r
Faux
\q
L'énergie potentielle de pesanteur 
\rv
peut être négative
\r
est toujours nulle au niveau de la mer
\rv
dépend de l'origine choisie
\r
diminue avec l'altitude
\q
Deux solides de masses $m_a$ et $m_b$ de vitesse $v_a = 2 v_b$ ont même énergie cinétique. Alors
\r
$m_b = 2 v_a$
\r
$m_b = 2 v_a$
\rv
autre réponse
  
\q
Dans un mouvement sans frottements, l'énergie mécanique se conserve forcément.
\r
Vrai
\rv
Faux
\q
Donner l'expression du joule en unités m, kg, s
