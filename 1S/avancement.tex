\q l'avancement maximal 
\r est égal à la quantité de matière initiale du réactif limitant 
\rv dépend des nombres stoechiométriques 
\rv dépend des quantités de matières initiales des réactifs 
\q On utilise une mole d'ions argent et 2 mol de cuivre lors de la transformation $$Cu+2Ag^+\longrightarrow Cu^{2+}+2Ag$$. Le réactif limitant est 
\r le cuivre 
\rv les ions argent 
\r les deux réactifs 
\q L'avancement est une variable 
\r qui diminue au cours du temps 
\r sans unité 
\rv qui s'exprime en mol 
\r qui s'exprime en gramme 
\q A la fin d'une transformation chimique totale
\r tous les réactifs ont forcément disparu
\r  seulement un réactif a disparu 
\rv le réactif limitant a disparu 
\rv l'avancement atteint sa valeur maximale
\rv l'avancement atteint sa valeur finale 
\q Dans un état final dans lequel les quantités de réactifs sont $3 - 3 x_m$ et $4 - 2 x_m$,  
\rv le réactif limitant est le réactif 1 
\r le réactif limitant est le réactif 2
\r $x_m=1$ 
\r $x_m=2$ mol 
\q 6 mol de dioxygène sont mélangés avec 10 mol d'aluminium, l'équation de la réaction est $2Al+3O_2\rightarrow Al_2O_3$\\ A l'état final 
\r il reste 3 mol de $O_2$ 
\rv il reste 6 mol de $Al$ 
\r 2 mol de Al ont été consommées
\rv il s'est formé 2 mol de $Al_2O_3$

\q L'avancement final est égal à l'avancement maximal 
\r toujours 
\r jamais 
\rv seulement pour des transformations totales 
\r seulement pour des transformations non totales
\q	
On introduit 1 mmol de poudre de zinc dans 100 mL d'une solution d'acide chlorhydrique de concentration C = 1,0 mol.L$^{-1}$. L'équation de la réaction modélisant la transformation chimique est : $$Zn_{(s)} + 2 H^+_{(aq)} = Zn^{2+}_{(aq)} + {H_2}_{(g)} $$.
En fin de transformation, on obtient 1 mmol d'ions $Zn^{2+}$ .
On donne : 	Masse molaire atomique du zinc : M(Zn) = 65 g.mol$^{-1}$ ;
Volume molaire des gaz dans les conditions de l'expérience : $V_M = 24$ L.mol$^{-1}$ ; 
\r
La masse de zinc introduite est de 65 g ;
\rv
La transformation est totale ;
\rv
Le volume de $H_2$ obtenu est de 24 mL.
\q On fait réagir une mole de dihydrogène avec 3 moles de dioxygène pour fabriquer de l'eau. A l'aide d'un tableau d'avancement, déterminer l'état final en supposant la réaction totale.
