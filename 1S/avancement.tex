\q la masse molaire 
\r dépend de l'état physique 
\r dépend de la température 
\r dépend du nombre d'entités 
\rv est la masse d'une mole d'entités 
\q La relation entre la masse $m$ d'un échantillon sa quantité de matière et sa masse molaire $M$ est
\rv
$m=nM$
\r
$n=M/m$
\q Une solution de diiode concentration C=50 mmol.L$^{-1}$, de volume V=50 mL contient
\r
25 mmol de diiode
\rv
autre réponse
\q On a M(Na) = 23 g.mol$^{-1}$, M(O) = 16 g.mol$^{-1}$. Une pastille de 4,0 g de soude NaOH contient 
\rv
100 mmol de soude
\rv
0.1 mol de soude
\r
0.01 mol de soude
\r 
10$^{-3}$ mol de soude
\q On veut prélever 0,200 mol de diiode dans une solution de concentration de concentration 2,0 mol.L$^{-1}$. Donner le protocole expérimental
\q	
On désire préparer une solution de concentration 1,0.10$^{-1}$ mol.L$^{-1}$ à partir  d'une solution mère de concentration 1,0 mol.L$^{-1}$.  On peut utiliser :
\r
Une fiole jaugée de 100 mL et une pipette jaugée de 1 mL ;
\r
Un bécher de 100 mL et une pipette jaugée de 10 mL ;
\rv
Une fiole jaugée de 50 mL et une pipette jaugée de 5 mL ;
\rv
Une fiole jaugée de 100 mL et une pipette jaugée de 10 mL.
\q	
On désire fabriquer 100 mL d'une solution de concentration   c = 0,020 mol.L$^{-1}$ en diluant 5 fois une solution mère. Choisir la verrerie nécessaire :
\r
becher de 100 mL ;		
\rv
pipette jaugée de 20 mL ;
\r
erlenmeyer de 100 mL ;	
\r
burette graduée de 50 mL ;
\rv
fiole jaugée de 100 mL ; 
\r
éprouvette graduée de 20 mL.
\q  On considère deux volumes identiques de dihydrogène et d'hélium (M(He)=4 g.mol$^{-1}$)
\rv ils renferment la même quantité de matière
\r ils ont la même masse
\rv la masse d'hélium est double
\r la quantité de matière d'hélium est quadruple 

\q Le volume molaire vaut 25 L.mol$^{-1}$. On dissout totalement 125 mL de gaz HCl dans 100 mL d'eau. On obtient une solution de concentration en HCl :
\rv 5.10$^{-2}$ mol.L$^{-1}$
\r 5.10$^{-3}$ mol.L$^{-1}$
\r autre réponse
 
\q
1 nm =
\rv
$10^{-9}$ m
\r
$10^{-6}$ m
\r
$10^{-3}$ m
\r
$10^{-12}$ m
\r
$10^{-15}$ m
\q
1 fm =
\r
$10^{-9}$ m
\r
$10^{-6}$ m
\r
$10^{-3}$ m
\r
$10^{-12}$ m
\rv
$10^{-15}$ m
\q
1 pm =
\r
$10^{-9}$ m
\r
$10^{-6}$ m
\r
$10^{-3}$ m
\rv
$10^{-12}$ m
\r
$10^{-15}$ m
\q
1 $\mu$m =
\r
$10^{-9}$ m
\rv
$10^{-6}$ m
\r
$10^{-3}$ m
\r
$10^{-12}$ m
\r
$10^{-15}$ m
\q
Convertir en mètre 0,023 $\mu$m
\rv
2,3. $10^{-8}$ m
\r
2,3. $10^{-6}$ m
\r
2,3. $10^{-4}$ m
\q
Le nombre de chiffres significatifs de 0,003450 m est
\rv
4
\r
6
\r
3
\r
5
\q
Le nombre de chiffres significatifs de 34,500 m est
\r
4
\r
6
\r
3
\rv
5
\q
Convertir une heure en millisecondes
\rv
$1 \U{h} = 3,6.10^6 \U{ms}$
\r
$1 \U{h} = 3,6.10^3 \U{ms}$
\r
$1 \U{h} = 3,6.10^{-6} \U{ms}$
\r
$1 \U{h} = 3,6.10^{-3} \U{ms}$
\q
La lumière se propage dans le vide à  la vitesse $c=3,0.10^8$ m/s.
Cela équivaut à  
\rv
$1,1.10^9$km/h
\r
$0,83.10^8$km/h
\q
La taille d'une petite molécule	est de l'ordre de $10^{-10}$m, soit
\rv
$0,1$nm
\r
100 nm
\r
10 nm	
\q
Le diamètre d'un cheveu	est de cinq dixièmes de micromètres, soit
\rv
500 nm
\r
500 pm
\rv
0,5$\mu$
\q 
Une seringue renferme 7 mL de sérum, soit
\rv
7 cm$^{3}$
\r
$7.10^{^-1}$cm$^{3}$
\r
$7.10^{^-3}$cm$^{3}$
\q
Un centimètre cube d'eau a pour masse
\rv
1 g
\r
100 g
\r
$10^{-3}$ g
\q
Le volume d'un ballon de foot est de l'ordre de 
\rv
5 L
\rv
5 $\U{dm}^{3}$
\r
500 mL
\q
Un cube de côté 2 cm a pour volume
\rv
$8$ cm$^{3}$
\rv
$8.10^{-3}$ L 
\rv
8 mL
\r
8 ML
\q
Estimer votre vitesse de marche en m/s. La convertir en km/h.
\q
 V = 340 mL peut s'écrire
\rv
$V =3,4.10^{-4}$m$^{3}$
\rv 
$V = 340.10^{-3}$L
\r
$V=34.10^{-6}$m$^{3}$
\q Convertir 12 g.L$^{-1}$ en mg.cm$^{-3}$ en détaillant le calcul


