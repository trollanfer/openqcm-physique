\q
Un champ électrique s'exprime en 
\r
m/V
\r
V.m
\rv
autre réponse
\q
Représenter les lignes du champ électrostatique dans un condensateur plan.
\q
Représenter les lignes de champ de la pesanteur dans cette salle.
\q
Représenter les lignes de champ gravitationnel de la Terre
\q
Préciser l'unité du champ gravitationnel
\r
N.kg
\r
V.m$^{-1}$
\rv
autre réponse
\q
On approche une sphère conductrice isolée, initialement neutre, d'un corps isolant chargé positivement fixe.
\r
rien ne se passe
\r
le métal est repoussé
\r
le métal est attiré et se colle
\rv
le métal est attiré, touche la sphère et est repoussé.
\q
On approche sans contact une sphère conductrice isolée, initialement neutre, d'un corps isolant chargé positivement fixe.
\r
Rien ne se passe
\r
des charges positives "apparaissent" en face du corps isolant
\rv 
des charges positives "apparaissent" à l'opposé du corps isolant
\q
On approche sans contact une sphère conductrice reliée à la terre, initialement neutre, d'un corps isolant chargé positivement fixe.
\r
rien ne se passe
\rv
la sphère se charge négativement
\r
la sphère se charge positivement
\q
Un corps chargé négativement a perdu des électrons
\rv
Faux
\r
Vrai
\q
Enoncer la loi de Coulomb
\q
Enoncer la loi de la gravitation universelle
\q
C'est parce que l'eau est conductrice qu'un filet d'eau peut être dévié par un corps chargé.
\rv
Faux
\r
Vrai
\q
La charge électrique totale d'un corps isolé est nécessairement constante
\rv
Vrai
\r
Faux


