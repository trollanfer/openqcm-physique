\q
Les foyers d'une lentille sont symétriques, donc $\overline{OF'}=\overline{OF}$
\rv
faux
\r
vrai
\q
Plus la vergence est grande, plus la distance focale est petite
\rv
vrai
\r
faux
\q
L'image d'un objet à l'infini se forme
\rv
sur le plan focal image
\r
sur le plan focal objet
\r
sur la lentille
\r
à l'infini
\q
Si on approche un objet d'une lentille, son image s'éloigne de celle-ci.
\rv
vrai
\r
faux
\q
Une image virtuelle
\rv
peut être vue à l'\oe il nu
\r
peut être observée sur un écran
\rv
se forme "avant" la lentille
\q
Si un objet est d'un côté d'une lentille convergente, son image est toujours de l'autre côté.
\rv
faux
\r
vrai
\q
Donner la définition du foyer objet d'une lentille.
\q2
Citer les trois rayons permettant de déterminer l'image d'un objet par une lentille, et leurs propriétés
\q
Lorsque l'objet se rapproche de l'appareil photo, pour faire la mise au point, on doit reculer la pellicule par rapport à l'objectif.
\rv
vrai
\r
faux
\q
Tous les objets devant un objectif peuvent donner une image nette par un appareil photo.
\rv
faux
\r
vrai
\q
Pour obtenir d'une diapositive une image nette sur un écran placé à quelques mètres, celle-ci doit se situer
\rv
juste avant le foyer objet de l'objectif
\r
au foyer objet de l'objectif
\r
juste après le foyer objet de l'objectif
\r
autre réponse
\q
Donner la définition du plan focal objet
\q
Donner la définition du plan focal image
\q
Un objet est placé entre le foyer objet et la lentille convergente. Son image est 
\rv
virtuelle
\rv
agrandie
\r
renversée
\q
Un objet est placé dans le plan focal objet d'une lentille convergente. Le grandissement correspondant vaut
\rv
$-\infty$
\r
$\infty$
\r
autre réponse
\q
Donner la relation de conjugaison de Descartes pour les lentilles minces. Faire un dessin précisant les notations.
\q
Un rayon incident passant par le foyer objet F
\r
ressort en passant par le foyer image F'
\rv
ressort parallèle à l'axe optique
\r
n'est pas dévié
\q
Une loupe de distance focale 5 cm a pour vergence 
\rv
20 $\delta$
\r
0.2 $\delta$
\rv
0,2 cm$^{-1}$
\q
On a \overline{OA} = -5 m et \overline{OA'} = 10 m. Quelle est la vergence de la lentille
?
\r
-0.3 $m^{-1}$
\rv
0.3 $m^{-1}$
\r
-0.3 $\delta$
\r
0.3 $\delta^{-1}$
\q 
Une lentille mince est convergente :
\r 
si elle est plus épaisse sur les bords que au centre
\rv
si elle est plus épaisse au centre que sur les bords

