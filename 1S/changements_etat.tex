\q
Calculer l'énergie thermique à donner à dix grammes d'eau liquide pour les faire passer de 298 à 308 K. On donne la capacité thermique de l'eau c = 4,18 kJ.kg$^{-1}$.K$^{-1}$
\rv
418 J
\r
418 kJ
\r
autre réponse
\q
Calculer l'énergie thermique à donner à un kilogramme d'eau solide à -1°C pour les faire passer entièrement à l'état liquide à 0°C. On donne la capacité thermique de la glace $c_1$ = 2,06 kJ.kg$^{-1}.K^{-1}$, de l'eau $c_2$ = 4,18 kJ.kg$^{-1}$.K$^{-1}$, et la chaleur latente de fusion de la glace L=333kJ.kg$^{-1}$.
\rv
335 kJ
\r
333 kJ
\r
337 kJ
\r
339 kJ


