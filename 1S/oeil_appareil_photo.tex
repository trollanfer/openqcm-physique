\q
Un \oe il au repos 
\rv
n'accommode pas
\r
a une vergence nulle
\rv
a une distance focale d'environ 2 cm.
\q
Lorsque l'objet se rapproche de l'\oe il, sa vergence augmente
\rv
vrai
\r
faux
\q
Lorsque l'objet se rapproche de l'appareil photo, pour faire la mise au point, on doit reculer la pellicule par rapport à l'objectif.
\rv
vrai
\r
faux
\q
Tous les objets devant un objectif peuvent donner une image nette par un appareil photo.
\rv
faux
\r
vrai
\q
Pour obtenir d'une diapositive une image nette sur un écran placé à quelques mètres, celle-ci doit se situer
\rv
juste avant le foyer objet de l'objectif
\r
au foyer objet de l'objectif
\r
juste après le foyer objet de l'objectif
\r
autre réponse
\q
Lorsqu'un objet s'approche de l'\oe il, celui-ci accommode en
\rv
diminuant sa distance focale
\rv
augmentant sa vergence
\r
dilatant l'iris
\q
Donner un ordre de grandeur de la distance focale du cristallin
\rv
2-3 cm
\r
2-3 mm
\r
autre réponse
\q
L'oeil peut-il faire une image nette de tous les objets? Justifier
\q
La lentille convergente présente dans l'oeil est appelée
\rv
cristallin
\r
pupille
\r
cornée
