\q
Donner la définition d'une désintégration radioactive, et un exemple
\q
Les noyaux très lourds se désintègrent principalement en émettant
\rv
des particules $\alpha$
\r
des électrons
\r
des positons
\q
La fission est une désintégration
\r
spontanée
\r
induite par le choc d'un proton sur le noyau
\rv
induite par le choc d'un neutron sur le noyau
\q
Au cours du temps, l'activité d'une source radioactive décroit obligatoirement
\rv
Vrai
\r
Faux
\q
La radioactivité $\beta^+$ correspond à l'émission
\rv
d'un positon
\r
d'un électron
\r
d'un proton
\r
d'un neutron
\q
Le radium$_{88}^{226}$Ra, émetteur $\alpha$, a pour noyau fils
\r
$_{84}^{224}$Po
\rv
$_{86}^{222}$Rn
\r
$_{88}^{222}$Ra
\q
Que signifie la phrase "Le coeur du réacteur de Fukushima a fusionné"?
\rv
Il a fondu
\r
Il y est survenu des réactions de fusion nucléaire
\q
On doit refroidir les coeurs de Fukushima longtemps après la catastrophe car des fissions s'y produisent encore
\r
Vrai
\rv
Faux
\q
Que valent A et B dans la formule $_A^Bn$? (n est un neutron)
\rv
A = 0, B = 1
\r
A = 1, B = 0
\q
Que valent A et B dans la formule $_A^B\beta^+$? ($\beta^+$ est un positon)
\r
A = 0, B = 1
\rv
A = 1, B = 0
\q
$_A^BU$
Que valent A et B ? (U est l'uranium)
\r
A = 235, B = 92
\rv
A = 92, B = 235
\q
Un kilogramme d'uranium peut libérer une énergie $E=mc^2=9.10^{16} J$.
\rv
Faux
\r
Vrai
\q
La réaction nucléaire donnant $_{28}^{60}Co$ à partir de $_{27}^{60}Co$ est une désintégration
\rv 
$\beta^-$
\r
$\alpha$
\r
$\beta^+$
\q
Le phosphore 30 ($_{15}^{30}P$) est radioactif $\beta^+$. Sa désintégration conduit à
\rv
$_{14}^{30}Si$
\r
$_{15}^{29}P$
\r
$_{16}^{30}S$
\q
Lors d'une réaction nucléaire, il y a toujours conservation du nombre de protons et de neutrons
\rv
Faux
\r
Vrai
\q
Lors d'une fusion ou fission nucléaire, 
\r
il y a toujours conservation de la masse.
\r
la masse des produits est plus grande que celle des réactifs
\rv
La masse des réactifs est plus grande que celle des produits


