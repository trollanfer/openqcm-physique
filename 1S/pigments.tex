\q
L'avancement de réaction
\rv
s'exprime en mol
\r
peut être négatif
\rv
est maximal en fin de réaction
\q
Si dans un état final, les quantités des deux réactifs sont, en mol, 

$n_1 = 4,0 - 2x_m$

$n_2 = 6,0 - 4x_m$
\rv
$x_m=$ 1,5 mol
\r
1 est limitant
\r
$x_m=$ 2 mol
\q
Un mélange stoechiométrique est un mélange o\`u les réactifs sont introduits en quantités égales.
\rv
Faux
\r
Vrai
\q
Après réaction totale d'un mélange stoechiométrique, 
\rv
les réactifs ont tous été entièrement consommés
\r
il s'est formé la même quantité de chacun des produits
\r
il a disparu la même quantité de chacun des réactifs
\q
Pour la réaction $3 A + 2 B \longrightarrow 3 C$, le mélange est stoechiométrique si initialement, leurs quantités de matière vérifient $3 n_A = 2 n_B$
\rv
Faux
\r
Vrai
%\q
%Citer la loi de Beer-Lambert
%\q
%L'absorbance d'une solution
%\rv
%n'a pas d'unité
%\rv
%est proportionnelle à sa concentration
%\rv
%dépend de la longueur d'onde utilisée
%\rv
%dépend de l'épaisseur de solution traversée
%\q
%Le coefficient d'extinction molaire s'exprime en 
%\rv
%$L.mol^{-1}.cm^{-1}$
%\r
%$L.mol^{-1}.cm$
%\r
%$L.mol.cm^{-1}$
\q
L'absorbance d'une solution diluée est proportionnelle à la longueur d'onde utilisée.
\rv
Faux
\r
Vrai
\q
Pour tracer une droite d'étalonnage pour un dosage spectrophotométrique, on doit mesurer l'absorbance de solutions
\rv
d'une même espèce, à concentrations différentes
\r
d'une même espèce, à longueurs d'ondes différentes. 


