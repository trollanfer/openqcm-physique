\q 
Une radiation verte a une longueur d'onde plus grande qu'une 
\r
radiation jaune
\rv 
radiation bleue
\r
radiation rouge
\q 
Une lampe a incandescence émet un spectre 
r\ 
de raies d'émission
\rv 
continu
\r 
d'absorption
\q
Exprimer une température en degrés Celsius en fonction de celle exprimée en Kelvin.
\q
Donner les longueurs d'ondes extrémales approximatives du visible.
\q
Citer trois sources de lumière respectivement:\begin{itemize}\item monochromatique \item de spectre de raies \item dont le spectre correspond approximativement à celui du corps noir \end{itemize}
\q
La longueur d'ondes des radiations ultraviolettes est
\rv
inférieure à 400 nm
\r
Supérieure à 800 nm
\q
La couleur d'un corps chauffé
\rv
dépend de l'ensemble des radiations qu'il émet
\r
dépend uniquement de la longueur d'onde de son maximum d'émission
\r
passe du jaune au rouge si sa température augmente
\q
Plus un corps est chaud, plus son spectre s'enrichit en radiations
\r
infrarouges
\rv
de courtes longueurs d'ondes
\r
de grandes longueurs d'ondes
\q
Les raies noires sur fond coloré sont caractéristiques
\r
d'un spectre d'émission
\rv
d'un élément chimique
\rv
de l'absorption de certains photons
\q
La loi de Wien permet
\rv
de mesurer la température de surface des étoiles
\r
de mesurer la masse des étoiles
\r
de mesurer la température interne des étoiles
\r
de déterminer la composition de surface du Soleil
\q
L'énergie transportée par un photon s'écrit $E=\f{h\lambda}{c}$
\r
Vrai
\rv
Faux
\q
On donne les énergies des premiers niveaux de l'atome de mercure : 
$E_0$ = -10,5 eV ; $E_1$ = -5,7 eV; $E_2$ = -5,5 eV; $E_3$ = -5 eV. Cet atome à l'état fondamental, peut absorber un photon d'énergie :
\rv
10,5 eV
\rv
5 eV
\r
5,7 eV
\rv
12 eV
\r
2 eV
\q
On donne les énergies des premiers niveaux de l'atome de mercure : 
$E_0$ = -10,5 eV ; $E_1$ = -5,7 eV; $E_2$ = -5,5 eV; $E_3$ = -5 eV. Cet atome se trouve à l'état d'énergie $E_2$. Donner toutes les émissions possibles, en précisant les énergies mises en jeu.
\q
Justifier qu'un rayon X est plus dangereux qu'une radiation UV.
\q
Lors d'une transition d'un niveau d'énergie égal à -1,5 eV à un niveau égal à -3,4 eV, l'énergie libérée est de
\rv
1,9 eV
\r
-4,9 eV
\r
5,1 eV
\r
-1.9 eV
\q
L'unité de l'énergie libérée au cours d'une transition atomique doit être en .... pour calculer la fréquence de l'onde émise en fonction de la fréquence du photon
\rv
En joule
\r
En électron Volt
\r
autre réponse
 \q
A quoi sont dues les raies sombres du soleil ?
\rv
La chromosphère
\r
La photosphère
\r
L'atmosphère terrestre
\q
Par définition, un spectre polychromatique est
\rv
un spectre à plusieurs raies
\r
un spectre continu, constitué de toutes les couleurs de l'arc en ciel.
\r
un spectre à une seule raie
