\q
Compléter l'équation de la réaction acido-basique suivante :
$HCO_2H_{(aq)}+ HO^{-}_{(aq)} =       …..     +  …….$
\q	
On désire préparer une solution A de concentration molaire 1,0.10$^{-1}$ mol.L$^{-1}$ à partir  d'une solution mère de concentration molaire 1,0 mol.L$^{-1}$  Pour cela on peut utiliser :
\r
Une fiole jaugée de 100 mL et une pipette jaugée de 1 mL ;
\r
Un bécher de 100 mL et une pipette jaugée de 10 mL ;
\rv
Une fiole jaugée de 50 mL et une pipette jaugée de 5 mL ;
\rv
Une fiole jaugée de 100 mL et une pipette jaugée de 10 mL.
\q	
 Donner la définition du pH
\q	
 Donner l'expression liant la conductivité $\sigma$ d'une solution aqueuse 
d'acide chlorhydrique, les concentrations molaires des ions $H_3O^+$ et $Cl^{-}$ 
présents dans la solution et les conductivités molaires ioniques  $\lambda$($H_3O^+$) et  $\lambda$($Cl^{-}$).
\q	
  Qu'est-ce qu'une transformation chimique limitée/équilibrée ?
\q
  	On dissout 0,10 mol d'ammoniac $NH_3$ gazeux dans de l'eau pure afin d'obtenir 200 mL de solution.
Le pH mesuré est égal à 11,2
\rv
  La réaction de l'ammoniac sur l'eau est : $NH_3(g) + H_2O(l) = NH_4^+ +  HO^{-}_{(aq)}$ ;
\r
  L'ammoniac joue ici le rôle d'un acide ;
\r
  Un pH-mètre permet de mesurer la conductance de la solution.
\q	
  Quels sont les couples acide/base dans lesquels on retrouve la molécule d'eau   ?
\q	
 On désire fabriquer 100 mL d'une solution de chlorure de potassium ($K+_{(aq)} + Cl^{-}_{(aq)}$) de concentration   c = 0,020 mol.L$^{-1}$ en diluant 5 fois une solution mère. Choisir la verrerie nécessaire :
\r
  becher de 100 mL ;		
\rv
    pipette jaugée de 20 mL ;
\r
  erlenmeyer de 100 mL ;	
\r
  	  burette graduée de 50 mL ;
\rv
  fiole jaugée de 100 mL ; 
\r
              	  éprouvette graduée de 20 mL.
\q
   	Le pH d'une solution d'acide nitreux $HNO_2$ de concentration c = 1,00.10$^{-2}$ mol.L$^{-1}$ vaut 2,7 :
\r  
L'équation de la réaction de l'acide nitreux avec l'eau s'écrit : 
$HNO_2 + HO^{-}_{(aq)}  = NO_3^{-} + H_2O_(l)$ ;
\rv
  L'acide nitreux appartient au couple acide base $HNO_2/NO_2^-$ ;
\rv
  L'acide nitreux réagit de façon limitée avec l'eau.
\q	
   Parmi les couples suivants, donner le (ou les) couple(s) acide/base ?
\r
 	$MnO_4^- / Mn^{2+}_{(aq)}$ ;
\rv
 	$CH_3COOH_{(aq)} / CH_3COO^{-}_{(aq)}$ ;
\rv
 	$H_2O_{(aq)} / HO^{-}_{(aq)}$ ;
\r
 	$S_2O_8^{2-}/ SO_4^{2-}$ ;
\rv
 	$CH_3NH_3^+ / CH_3NH_2$.

\q	
   On considère un volume V = 100 mL de solution d'acide chlorhydrique $(H_3O^+_{(aq)} + Cl^-_{(aq)})$, à 25 °C, de concentration c en soluté apporté (c = 1,0 . 10$^{-3}$ mol.L$^{-1}$). Le pH de cette solution est égal à 3. 
\rv
  La réaction du chlorure d'hydrogène $HCl_{(aq)}$ avec l'eau est une transformation chimique totale ;
\r
  L'avancement maximal de la réaction du chlorure d'hydrogène $HCl_{(aq)}$ avec l'eau est xmax = 1,0 . 10$^{-3}$ mol;
  \rv
  L'avancement maximal de la réaction du chlorure d'hydrogène $HCl_{(aq)}$ avec l'eau est xmax = 1,0 . 10$^{-4}$ mol;
 \q	
Soient trois solutions $S_1$, $S_2$ et $S_3$ dont on connaît les propriétés suivantes à 25° C:
Pour $S_1$, $[H_3O^+]$ = 4,0 .10$^{-5}$ mol.L$^{-1}$ ;
Pour $S_2$, pH = 5 ;
Pour $S_3$, 100 mL contiennent 6,0.10$^{-6}$ mol d'ions $H_3O^+$.
A partir de la détermination de $[H_3O^+]$, indiquer si :
\rv
Toutes les solutions sont acides ;
\r
Les trois solutions ont le même pH ;
\r
$S_2$ est la plus acide.
\q	
On dispose d'une solution d'un acide HA de concentration en soluté apporté $c_a$ = 1,0.10$^{-2}$  mol.L$^{-1}$. Son pH est égal à  2,5.
\rv
La réaction de l'acide avec l'eau est limitée 
\r
La réaction de l'acide avec l'eau est totale
\q	
Exprimer la constante d'acidité Ka du couple $NH_4^+/ NH_3$.
\q	Exprimer la constante d'acidité Ka du couple $HCOOH/HCOO^{-}$.
   \q	
On prépare une solution $S_2$ d'acide formique (ou acide méthanoïque) de concentration        $c_2 = 5,0 . 10^{-3}$ mol.L$^{-1}$ à partir d'une solution $S_1$ d'acide formique de concentration $c_1 =1,0 . 10^{-1}$ mol.L$^{-1}$ 
\r
Pour la dilution on doit utiliser une éprouvette de 5 mL et une fiole jaugée de 100 mL ;
\rv 
La réaction de l'acide avec l'eau s'écrit :
$HCOOH_{(aq)}+H_2O(l) = HCOO^{-} _{(aq)}+ H_3O^+ _{(aq)}$ ;
\rv
La constante d'équilibre de la réaction de l'acide formique dans l'eau s'écrit :
$Ka=\dfrac{[HCOO^{-}][H_3O^+]}{[HCOOH]}$.
\q	
Donner l'équation de la réaction entre l'acide chlorhydrique ($H_3O^+_{(aq)} + Cl^{-}_{(aq)}$) et une solution d'hydroxyde de sodium ($Na^{+}_{(aq)} + HO^{-}_{(aq)}$).
  \q	
On considère un volume V = 100 mL de solution d'acide chlorhydrique $(H_3O^+_{(aq)} + Cl^-_{(aq)})$, à 25 °C, de concentration c en soluté apporté (c = 1,0 . 10$^{-3}$ mol.L$^{-1}$). Le pH de cette solution est égal à 3. 
On donne Ke = 1,0 . 10$^{-14}$ à 25 °C.
\rv
La réaction du chlorure d'hydrogène $HCl_{(aq)}$ avec l'eau est une transformation chimique totale ;
\r
L'avancement maximal de la réaction du chlorure d'hydrogène $HCl_{(aq)}$ avec l'eau est $x_m$ = 1,0 . 10$^{-3}$ mol;
\rv
Dans cette solution, $[HO^{-}]_f$ = 1,0 . 10$^{-11}$ mol.L$^{-1}$   
 \q	
On considère une solution aqueuse contenant de l'acide éthanoïque.
Le pKa du couple $CH_3COOH_{(aq)}/ CH_3COO^{-}_{(aq)}$  vaut 4,8 ;
\rv
La forme $CH_3COO^{-}_{(aq)}$  prédomine  pour une solution de pH = 6,5 ;
\r
Les formes $CH_3COOH_{(aq)}$ et $CH_3COO^{-}_{(aq)}$ sont présentes en quantités égales pour une solution de pH=7 ;
\r
Si le pH de la solution vaut 8, celle-ci ne contient pas d'ions $H_3O^+_{(aq)}$.
\q
On dispose d'un verre de jus d'orange (pH = 3,0) et d'une tasse de café (pH = 6,0). 
\rv
Le jus d'orange contient des ions $HO^{-}_{(aq)}$ ; 
\r
La concentration en ions $H_3O^+_{(aq)}$  est plus importante dans le café ;
\rv
Les concentrations en ions $H_3O^+_{(aq)}$ sont dans un rapport de  1000.
\q
Rappeler l'expression et la valeur du produit ionique de l'eau.
 \q	
A 25° C, la concentration molaire des ions oxonium d'une solution est 1,0.10$^{-3}$ mol.L$^{-1}$ . 
On donne Ke = 1,0 . 10$^{-14}$ 
\rv
pH = 3,0 ;
\rv La concentration molaire des ions hydroxyde est 1,0.10$^{-11}$ mol.L$^{-1}$ ;
\r
La solution est basique.

\q	
On donne le produit ionique de l'eau : Ke = 1,0.10$^{-14}$ à 25°C.
\rv
Pour un couple HA/A-, l'espèce HA est prédominante lorsque pH est inférieur au pKa du couple ;
\rv
Plus la constante d'acidité Ka d'un couple acide/base est grande, plus l'acide est dissocié dans l'eau ;
\r
On peut écrire [$H_3O^+]_f = [HO^{-}]_f$ = 1,0.10$^{-7}$ mol.L$^{-1}$ quelle que soit la solution aqueuse étudiée ;
\r	
Le pH d'une solution aqueuse d'hydroxyde de sodium dans laquelle la concentration molaire en ions hydroxyde est 1,0.10$^{-3}$ mol.L$^{-1}$ a pour valeur pH = 3 
\q	
La réaction d'équation $2 H_2O(l) = H_3O^+_{(aq)}  + HO^{-}_{(aq)}$ présente la ou les propriétés suivantes :
\r
Elle ne se produit que dans l'eau pure ; 
\r
$[H_3O^+] = [HO^{-}]$  dans toutes les solutions où elle se produit ;
\rv
Elle a un avancement très faible dans l'eau pure.
 \q	
Soit la réaction d'équation :	$2 H_2O(l) = H_3O^+_{(aq)}  +  HO^{-}_{(aq)}.$
\rv
Elle correspond à une réaction acido-basique ;
\rv
La constante d'équilibre associée est appelée Ke ; 
\rv
Elle est appelée réaction d'autoprotolyse de l'eau ;
\r
Elle n'a lieu que dans l'eau pure.
   \q	On considère la réaction :	 $AH_{(aq)} + H_2O(l) = A^-_{(aq)} + H_3O^+_{(aq)}$
\rv
Il s'agit d'une réaction acido-basique ;
\r 	
A l'équilibre : pH = pKa + log $\dfrac{[AH]}{[A^-]}$  ;
\r
Si pH < pKa, l'espèce $A^-_{(aq)}$ est prédominante.
\q
On dissout 0,10 mol d'ammoniac $NH_3(g)$ dans de l'eau distillée de façon à obtenir 200 mL de solution.
Le pH de la solution est de 11,4 :
On donne :    Ke=10$^{-14}$ ;     pKa ($NH_4^+/NH_3$)=  9,2 .
\rv
La réaction de l'ammoniac sur l'eau est modélisée par $NH_{3(aq)} + H_2O(l) = NH_4^+ + HO^{-}_{(aq)}$ ;
\r
L'ammoniac est une espèce acide ;
\q	
Un acide AH réagit totalement avec l'eau. Une solution aqueuse de cet acide a une concentration égale à 1,0 . 10$^{-3}$ mol.L$^{-1}$ 
\rv
Le pH de cette solution est 3,0 ;
\r
Lorsqu'on dilue cette solution, le pH diminue ;
\r
Cette solution étant acide elle ne contient pas d'ions $HO^{-}$.

  
