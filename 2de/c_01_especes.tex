\q
l'alcool à 70 degrés, constitué d'eau de d'éthanol, est
\rv
un mélange
\r
un corps pur
\r
une espèce chimique
\q
La masse volumique a pour expression
\rv
$\rho=\f{m}{V}$
\r
$\rho=\f{V}{m}$
\r
$\rho=V\times m$
\q
La masse volumique de l'eau est 
\rv
$1\U{kg.m^{-3}}$
\rv
$1\U{g.cm^{-3}}$
\r
$1\U{g.m^{-3}}$
\r
$1\U{g.L^{-1}}$
\q
L'unité de la densité est 
\r
le$ \U{kg.L^{-1}} $
\rv
sans unité
\r
le$ \U{g.L^{-1}} $
\q
Deux liquides sont miscibles. Alors, 
\rv
il n'existe qu'une seule phase
\r
le moins dense surnage
\r
le plus dense surnage
\q
Deux liquides sont non miscibles. Alors, 
\r
il n'existe qu'une seule phase
\rv
le moins dense surnage
\r
le plus dense surnage
\q
Lors d'une extraction liquide-liquide, l'espèce chimique à isoler doit être
\rv
plus soluble dans le solvant extracteur
\r
plus soluble dans le solvant initial
\r
insoluble dans le solvant extracteur
\r
\q
Dans une chromatographie, l'éluant permet de 
\rv
séparer les espèces du mélange
\r
révéler le chromatogramme
\r
diluer la solution
\q
Donner la définition du rapport frontal. Faire un dessin.
\q
Le rapport frontal 
\rv
est toujours compris entre 0 et 1
\r
peut être négatif
\r
peut être supérieur à 1
\q
Le rapport frontal
\rv
n'a pas d'unité
\r
s'exprime en mètre
\r
s'exprime en centimètre
\q 
La fusion correspond au passage
\rv
de l'état solide à l'état liquide
\r
de l'état solide à l'état gazeux
\r
de l'état liquide à l'état gazeux
\r
de l'état liquide à l'état solide
\q 
La sublimation correspond au passage
\r
de l'état solide à l'état liquide
\rv
de l'état solide à l'état gazeux
\r
de l'état liquide à l'état gazeux
\r
de l'état liquide à l'état solide
\q 
La condensation correspond au passage
\r
de l'état gazeux à l'état liquide
\r
de l'état solide à l'état gazeux
\r
de l'état liquide à l'état gazeux
\rv
de l'état gazeux à l'état solide
\q
Donner la définition d'un corps pur.
\q
Une extraction liquide-liquide se réalise avec
\rv
une ampoule à décanter
\r
un erlenmeyer
\r
une éprouvette

\q
Donner le nom d'une technique permettant d'extraire une espèce chimique volatile.

\r
dioxyde de carbone
\rv
eau
\r
chlorure de sodium
\q
Une masse volumique s'exprime en $\U{g.cm^{-2}}$.
\r
Vrai
\rv
Faux
\q
Citer trois caractéristiques physiques permettant d'identifier une espèce chimique
\q
La densité a pour définition $d=\f{\rho\e{échantillon}}{\rho\e{eau}}$
\rv
Vrai
\r
Faux
\q
La densité a le $\kgpmcub$ pour unité
\rv
Faux
\r
Vrai
\q
Une espèce chimique A (température d'ébullition 86°C) est en solution dans l'eau. Peut-on appliquer la technique d'entrainement à la vapeur pour séparer A ? Justifier.
\q
Une espèce chimique a pour densité 1,2. Donner sa masse volumique, en précisant les unités.
\q
Schématiser le montage d'entrainement à la vapeur
\q
Schématiser une chromatographie sur couche mince