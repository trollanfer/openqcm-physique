\desc{seconde}{lois de Newton}
\q
Dans le référentiel géocentrique
\rv
la Terre tourne sur elle même
\r
la Terre est immobile
\r
le Soleil est immobile
\rv
les étoiles sont immobiles
\q
Pour étudier le mouvement d'un satellite, on doit se placer dans le référentiel
\rv
géocentrique
\r
terrestre
\r
héliocentrique
\q
Pour étudier le mouvement de la Terre, on doit se placer dans le référentiel
\r
géocentrique
\r
terrestre
\rv
héliocentrique
\q
La Terre exerce une force sur vous. En exercez vous une sur elle?
\rv
oui
\r
non
\q
Dans le référentiel héliocentrique, les planètes sont
\rv
toutes en mouvement
\r
fixes pour certaine(s), en mouvement pour d'autres
\q
Dans le référentiel géocentrique, le Soleil 
\r
est fixe
\rv
tourne en une année
\r
tourne en un jour
\q
Dans le référentiel terrestre, le Soleil 
\r
est fixe
\r
accomplit une rotation en une année
\rv
accomplit une rotation en un jour
\q
Dessiner la trajectoire de la Terre dans le référentiel géocentrique.
\q
Dessiner la trajectoire de la Terre dans le référentiel héliocentrique.
 \q
        Pour décrire le mouvement d'un objet, il faut toujours définir 
  \r
       la vitesse
  \r
       son poids
  \rv
       un référentiel
  \r
       l'accélération
  \q
  Le référentiel lié à la surface de la Terre est le référentiel 
  \rv
       terrestre
  \r
       héliocentrique
  \r
       géocentrique
  \r
       saturnocentrique
  \q
  Le référentiel lié au centre de la Terre est le référentiel 
  \r
       terrestre
  \r
       saturnocentrique
  \rv
       géocentrique
  \r
       héliocentrique
  \q
  Le référentiel lié au centre du Soleil est le référentiel 
  \r
       saturnocentrique
  \rv
       héliocentrique
  \r
       terrestre
  \r
       géocentrique
\q
Dans le référentiel géocentrique
\rv
la Terre tourne sur elle même
\r
la Terre est immobile
\r
le Soleil est immobile
\rv
les étoiles sont immobiles
\q
Pour étudier le mouvement d'un satellite, on doit se placer dans le référentiel
\rv
géocentrique
\r
terrestre
\r
héliocentrique
\q
Pour étudier le mouvement de la Terre, on doit se placer dans le référentiel
\r
géocentrique
\r
terrestre
\rv
héliocentrique
\q
La Terre exerce une force sur vous. En exercez vous une sur elle?
\rv
oui
\r
non
\q
Si $F_1$ est la force exercée par la Terre sur la Lune et $F_2$ la force exercée par la Lune sur la Terre, alors
\rv
$F_1=F_2$
\r
$F_1<F_2$
\r
$F_1>F_2$
\q
Le force gravitationnelle échangée entre deux corps $m_1$ et $m_2$ distants de $d$ s'écrit
\rv
$F=G\dfrac{m_1 m_2}{d^{2}}$
\r
$F=G\dfrac{d}{m_1 m_2}$
\r
$F=G\dfrac{m_1 m_2}{d}$
\q
Un astronaute sur la Lune a
\rv
un poids plus faible
\r
une masse plus faible
\r
un poids inchangé
\rv
une masse inchangée
\q
La Lune n'est en interaction gravitationnelle qu'avec la Terre
\rv
faux
\r
vrai
\q
Quel est le poids sur Terre d'un objet de masse 60 kg?
\r
environ 60 N
\rv
environ 600 N
\r
environ $6.10^5$ N
\q
Le poids d'un objet est indépendant de sa localisation sur Terre ou dans l'espace.
\rv
Faux
\r
Vrai
\q
Un objet de poids 10 N exerce sur le centre de la Terre
\r    
une force répulsive de 10 N
\rv
une force attractive de 10 N
\r
une force nulle