\q
Justifier que la lumière se propage dans le vide sur un exemple
\q
La valeur de la vitesse de propagation de la lumière dans le vide est voisine de :
\r
	2,0.10$^{8}$ m.s$^{-1}$.
\rv
	3,0.10$^{8}$ m.s$^{-1}$.
\r
3,0.10$^{8}$ km.s$^{-1}$.
\q
Le rayon de l'atome d'hydrogène est 0,053 nm. L'ordre de grandeur correspondant est: 
\rv
$10^{-10}$ m
\r
$10^{-6}$ m
\r
$10^{-15}$ m
\q La distance Terre-Soleil est de 150 millions de kilomètres. L'ordre de grandeur correspondant est: 
\r
$10^{10}$ m
\r
$10^{9}$ m
\rv
$10^{8}$ m
\q
1 mètre représente: 
\rv
$10^{-3}$ km
\r
$10^{12}$ µm
\q
Regarder loin, c'est aussi regarder dans le passé. Justifier.
\q
Un mètre représente
\rv
$10^{6}$ µm
\rv
$10^{-3}$ mm
\r
$10^{-3}$ km
\q
Si une étoile située à 1000 a.l. explose, nous le saurons dans mille ans.
\r
faux
\rv
vrai
\q
Une seconde-lumière correspond à
\rv
une distance de $3.10^{8}$ m
\r
une durée de $3.10^{8}$ s
\r
autre réponse
\q
Entre le noyau et les électrons d'un atome, il y a 
\r
des molécules 
\rv
du vide
\r
de l'air     
\q     
Entre le Soleil et les planètes du système solaire, il y a :   
\rv
surtout du vide   
\r
de l'air   
\r
des galaxies
\q  
Une galaxie
\r
est un regroupement de planètes autour d'une étoile   
\rv
est un regroupement d'étoiles dont certaines sont accompagnées de planètes
\rv
a une structure lacunaire     
\q      
Dans le vide, la lumière: 
\rv
se propage en ligne droite 
\r
ne se propage pas  
\rv
se propage avec une vitesse constante   
\q
Donner la définition de l'année lumière.
\q       
La lumière met 8 min 20 s pour nous parvenir du Soleil. Quelle est la distance entre la Terre  et le Soleil?         
\r
600000 m
\rv
$150.10^6$ km 
\r
$1,5.10^{11}$ km
\q
La distance entre le Soleil et Vénus est de $108,2.10^6$ km. Cette distance s'écrit aussi:                 
\r
$1,082.10^8$ m
\rv
$1,082.10^{11}$ m
\r
$108,2.10^{10}$ m
\q 
Le rayon du Soleil est d'environ $6,96.10^8$ m. L'ordre de grandeur de cette distance   
est:                   
\r 
$10^8$ m 
\rv
$10^9$ m 
\r
$7.10^8$ m 
\q
Le diamètre d'un virus du sida est de 105 nm. L'ordre de grandeur de ce diamètre   
est:           
\r
$10^{9}$ m
\rv
$10^{-7}$ m
\r
$10^{-11}$ m
\q
L'année lumière est une unité de temps.
\rv
Faux
\r
Vrai