%%% scanner p 161 hachette
%\q
%\fig{periodeUn}
%\rv
%Ce signal est périodique
%\r
%Sa période vaut environ 0,4 ms
%\rv
%Sa période vaut environ 0,8 ms
\q
Un pouls mesuré à 120 pulsation par minute correspond à une fréquence cardiaque de 
\rv
2 Hz
\r
120 Hz
\r
240 Hz
\r
0,5 Hz
\q
Donner deux exemples de phénomènes périodiques dont la période est proche de la seconde
\q
Donner la définition de "périodique"
\q
La fréquence d'un phénomène est
\rv
le nombre de périodes se déroulant en une seconde
\rv
l'inverse de la période, celle ci étant exprimée en seconde.
\r
la durée au bout de laquelle le phénomène se répète.
\q
L'ordre de grandeur de la fréquence des battements de votre c\oe ur est
\rv
1 Hz
\r
10 Hz
\r
100 Hz
\rv
60 battements par minute
\q
Un moteur tourne à 3000 tours par minute. Sa fréquence de rotation est de
\rv
500 Hz
\r
18000 Hz
\r
0,002 Hz
\q
Le secteur EDF a une fréquence de 50 Hz. Sa période est
\rv
20 ms
\r
50 ms
\rv
0,02 s
\q
Dessiner un signal périodique, préciser sur le dessin une période.
\q
Un balancier passe de droite à gauche en 1 seconde.
Quelle est sa fréquence?
\rv
0.5 Hz
\r
1 Hz
\r
2 Hz
\q
Si T = 4 $µ$s, alors  f = 250 Hz
\r
Vrai
\rv
Faux
\q
La fréquence d'un phénomène périodique est le nombre de périodes que l'on peut inscrire dans une durée d'une unité de temps (une seconde). 
\rv
Vrai
\r
Faux
\q 
Une onde sonore peut se propager dans le vide
\rv
Faux
\r
Vrai
\q
Une onde sonore peut se propager dans un solide
\r
Faux
\rv
Vrai
\q
Une onde sonore de fréquence 30 kHz fait partie
\rv
des ultrasons
\r
du domaine audible
\r
des infrasons
\q
Une onde sonore de fréquence 3 kHz fait partie
\r
des ultrasons
\rv
du domaine audible
\r
des infrasons
\q
Une onde sonore de fréquence 30 mHz fait partie
\r
des ultrasons
\r
du domaine audible
\rv
des infrasons


