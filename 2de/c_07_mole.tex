          \q
Dans une mole d'atomes, il y a 
\rv
$6,02.10^{23}$ atomes
\rv
$602.10^{21}$ atomes
\r
$6,02.10^{-23}$ atomes
\q
La constante d'Avogadro vaut :
\rv
$6,02.10^{23}$ mol$^{-1}$
\r 
$6,02.10^{23}$
\r
$6,02.10^{23}$ mol
\q 
Deux masses différentes d'espèces chimiques différentes peuvent renfermer la même quantité de matière.
\rv
Vrai
\r
Faux
\q
La masse molaire du glucose ($C_6H_{12}O_6$) vaut
180 g.mol.
\rv
Faux
\r
Vrai
\q
Donner valeur de la constante d'Avogadro et son unité

\q
1 mole d'atomes de fer contient 
\rv
       6,02 .10$^{23}$ atomes de fer
\rv
       60,2 .10$^{22}$ atomes de fer
\r
       6,02 .10$^{-22}$ atomes de fer
\r
       6,02 .10$^{-23}$ atomes de fer
\q
La masse molaire du diiode $I_2$ ($_{53}^{127}$I) est
\r
       63,45 g mol$^{-1}$
\r
       126,9 g mol$^{-1}$
\r
       16 104 g mol$^{-1}$
\rv
       253,8 g mol$^{-1}$
\q
La masse molaire du tétrafluorure d'uranium UF4 est 
\rv
       1028 g mol$^{-1}$?
\rv
       238 g mol$^{-1}$?
`rv
       314 g mol$^{-1}$???
\rv
       257 g mol$^{-1}$???
\q
La masse molaire de l'ion chlorure Cl$^{-}$ est 
\r
       36,5 g mol$^{-1}$
\r
       34,5 g mol$^{-1}$
\rv
       35,5 g mol$^{-1}$
\q
La masse molaire de l'ion nitrate NO$_3^-$ est 
\r
       29 g mol$^{-1}$
\r
       30 g mol$^{-1}$
\rv
       62 g mol$^{-1}$
\r
       61 g mol$^{-1}$
\q
La relation liant la quantité de matière n à la masse m de l'échantillon de masse molaire M est: 
\rv
       n = m / M
\r
       m = n / M
\rv
       m = n×M
\r
       n = m×M
\q
La quantité de matière contenue dans 10 g de diiode est
\r
       2,54 .10$^{1}$ mol
\rv
       3,9 .10$^{-2}$ mol
\r
       2,54 .10$^{3}$ mol
\q
La masse de 0,25 mol de tétrafluorure d'uranium est 
\r
       7,9 .10$^{-}$4 mol
\r
       78,5 kg
\rv
       78,5 g
\r
       1256 g
\q
Le volume molaire d'un gaz dépend 
\r
       de sa nature
\rv
       de sa température
\r
       de la composition du gaz
\q
La relation liant la quantité n d'un gaz contenue dans un volume V est 
(Vm est le volume molaire)
\rv
       V = n × Vm
\r
       V = n / Vm
\r
       n = V × Vm
\rv
       n = V / Vm
\q
La quantité de matière contenue dans 10 L de dioxygène est 
(Vm = 25 L mol$^{-1}$)
\rv
       0,4 mol
\r
       250 mol
\r
       2,5 mol
\q
Le volume occupé par 0,33 mol de dihydrogène est 
(Vm = 25 L mol$^{-1}$)
\r
       75,8 L
\r
       1,32 .10$^{-}$2 L
\rv
       8,25 L