\q
Donner un exemple de milieu dispersif, et une application
   \q
       La lumière blanche :
\r
	a un spectre discontinu.
\r
	peut être émise par un laser.
\rv
	est une lumière polychromatique.

\q
Plus une source est froide, plus elle émet des radiations de courtes longueurs d'onde.
\rv
Faux
\r
Vrai
\q
Une espèce chimique gazeuse sous faible pression est éclairée en lumière blanche. Le spectre obtenu est :
\rv
caractéristique de cette espèce
\rv
constitué de raies noires sur fond coloré
\r
appelé spectre d'absorption
%\q
%Voici le spectre d'absorption de l'hydrogène :
%%%%%%%%%%%%%%%FIGURE
%\fig{spectresAbsoptionH}
%Dessiner son spectre d'émission
%\q
%\fig{spectresAmpoules}
%On a superposé le spectre de deux ampoules A et B
%\rv
%A est plus chaude que B
%\rv 
%A est plus blanche que B
%\r
%A est une ampoule à décharge
\q
Pourquoi éclaire-t-on les rues avec des lampes à décharge et non des lampes halogène?
\rv
Car elles consomment moins d'énergie
\r
Car leurs couleurs sont plus agréables
\rv
Car elles chauffent moins à puissance égale
\q
Le profil spectral d'une étoile permet d'accéder à 
\r
sa température interne
\rv
sa température de surface
\r
sa masse
\rv
la composition chimique de sa partie externe
\q 
Une étoile est composée majoritairement
\rv
d'hydrogène et hélium
\r
d'hydrogène et oxygène
\r
d'hydrogène et carbone
\q
Compléter: la lumière blanche d'une lampe à incandescence est composée .......... radiations colorées.
\rv
   d'une infinité de
\r
    d'aucune
\r
      d'un nombre restreint de
\r
    d'une unique
     \q
	Une lumière monochromatique est composée...... radiation(s) colorée(s).
\r
       d'un nombre restreint 
\r
    d'une infinité 
\rv
    d'une seule 
\r
    d'aucune 
\q
	Une lumière polychromatique peut être composée......  radiation(s) colorée(s). 
\rv
    d'un nombre restreint de
\r
    d'aucune 
\r
    d'une seule 
\rv
    d'une infinité de
\q
	L'oeil humain est sensible aux lumières dont les longueurs d'ondes sont comprises entre:
\rv
    0,4 et 0,8 µm
\r
    0,4 et 0,8 pm
\rv
    400 et 800 nm
\r
    400 et 800 µm
     \q
       Pour réaliser le spectre d'une lumière, on peut utiliser:
     \rv
       un prisme
       \rv
       un réseau
       \r
       un fil
       \r
       une lentille
\q
Le spectre de la lumière émise par une lampe à incandescence est un spectre
\r
       de raies d'émission
\r
       de bandes d'absorption
\r
       de raies d'absorption
\rv
       d'émission continu
\q
Le spectre de la lumière émise par une lampe à vapeur à vapeur de sodium est un spectre 
\r
       de raies d'absorption
\rv
       de raies d'émission
\r
       d'émission continu
\r
       de bandes d'absorption
\q
Le spectre de la lumière blanche après avoir traversée un gaz monoatomique est un spectre
\r
       de bandes d'absorption
\r
       d'émission continu
\rv
       de raies d'absorption
\r
       de raies d'émission
\q
Le spectre de la lumière blanche après avoir traversée une solution colorée est un spectre
\r
       de raies d'émission
\r
       de raies d'absorption
\rv
       de bandes d'absorption
\r
       d'émission continu
\q
Une lampe à incandescence brille fortement. On diminue progressivement la tension d'alimentation. Comment évolue le spectre de la lumière émise ?
\rv
       il s'appauvrit de radiations colorées de longueurs d'ondes courtes.
\r
       il s'appauvrit de radiations colorées dans le domaine du rouge.
\rv
       il s'enrichit de radiations colorées dans le domaine du rouge.
\r
       il s'enrichit de radiations colorées de longueurs d'ondes courtes.
\q
Le spectre d'une étoile permet
\r
       de déterminer sa taille
\r
       de déterminer la composition de son centre
\rv
       de déterminer la composition de son atmosphère.

\q
Lorsque la température augmente, la lumière émise par un corps chauffé
\rv
s'enrichit vers le bleu
\r
s'enrichit vers le rouge
\rv
devient plus intense
\r
devient un spectre de raies
\q
Le fond continu du spectre d'une étoile donne des renseignements sur
\r
la température au coeur de l'étoile
\rv
la température de surface de l'étoile
\r
sa composition chimique en surface
\q
Les raies du spectre d'une étoile donnent des renseignements sur
\r
la température au coeur de l'étoile
\r
la température de surface de l'étoile
\rv
sa composition chimique en surface
\q
Les raies du spectre d'absorption d'un gaz
\rv
correspondent à ses raies d'émission
\r
se déplacent suivant la température
\rv
sont noires sur fond coloré
\q
Les raies sombres du spectre solaire mettent en évidence
\rv
la présence de certains gaz en surface du Soleil
\r
l'absence de certains gaz en surface du Soleil
\r
la température de surface du Soleil     

	\q
  Après décomposition de la lumière blanche, on obtient
	  \r
         un spectre de raies
	     \rv
       un spectre continu
