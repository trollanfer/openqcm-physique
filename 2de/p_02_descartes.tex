%\desc{seconde}{Descartes}
%\q
%En modifiant l'inclinaison du rayon se propageant dans le milieu 1, dans quel cas est-il %possible d'obtenir une réflexion totale ? 
%dessin à scanner
%\rv
%cas 1
%\r
%cas 2
%\includegraphics[height=20 mm,angle=0]{chapitres/fig/descartes1.jpg}
%
%\q
%    	L'énoncé de la 1ère loi de Descartes est:
%\r
%  le rayon réfracté est contenu dans le plan de la surface de séparation des %2 milieux transparents.
%\rv
%    le rayon réfracté est contenu dans le plan d'incidence.
%\r
%    le rayon réfracté est parallèle au plan d'incidence.
%\r
%    le rayon réfracté est perpendiculaire au plan d'incidence.
\q
La seconde loi de Descartes peut s'écrire
\r
	     $n_1. sin(i_2) = n_2. sin(i_1)$
\rv
    $ n_1  sin(i_1) = n_2  sin(i_2)$
\r
	      $n_1/sin(i_1) = n_2/sin(i_2)$
\rv
	      $sin(i_1)/n_2 = sin(i_2)/n_1$
\q	      
	Le rayon d'un faisceau d'une lumière monochromatique est dirigé sur une lame de verre d'indice de réfraction 1,47. L'angle d'incidence est de 40° et le milieu d'incidence est l'air. L'angle de réfraction vaut:
\r
	      71°
\r
	      40°
\rv
	      26°
\r
	      51°
\q
	L'indice de réfraction n caractérisant un milieu transparent est donné par la relation (on désigne par c et v les vitesses respectives de la lumière dans le vide et dans le milieu considéré):
\r
	      n = v/c
\rv
      n = c/v
\r
	      n = c + v
\r
	      n = c.v
\q
	L'indice de réfraction de l'eau est de 1,33. Quelle est la vitesse de propagation de la lumière dans ce milieu ?
\r
	      4,00 .10$^{8}$ m.s$^{-1}$
\r
	      4,43 .10$^{8}$ m.s$^{-1}$
\r
	      2,25 .10$^{8}$ m.s$^{-1}$
\q
Le changement de direction d'un rayon lumineux changeant de milieu est appelé
\rv
réfraction
\r
réflexion
\r
dispersion
%\q
%\fig{descartes_angles p 49 hachette} 
%\rv
%On a $n_1<n_2$
%\r
%L'angle d'incidence est $\beta$
%\r
%L'angle de réfraction est $\alpha$
\q
Les infrarouges ont des longueurs d'onde 
\rv
supérieures à 800 nm
\r
inférieures à 800 nm
\r
supérieures à 400 nm
\r
inférieures à 400 nm
\q
Les ultraviolets ont des longueurs d'onde 
\r
supérieures à 800 nm
\r
inférieures à 800 nm
\r
supérieures à 400 nm
\rv
inférieures à 400 nm
\q
La loi de Descartes relative à la réflexion peut s'écrire
$\f{n_1}{n_2}=\f{sin(i_1)}{sin(i_2)}$
\rv
Faux
\r
Vrai


%\q
%Sur la figure 1, l'angle d'incidence est l'angle: $\alpha ,\beta, \gamma$ 

\q
Lorsqu'un rayon lumineux arrive sur un dioptre, il y a toujours un rayon réfracté.
\rv
Faux
\r
Vrai
\q
Un rayon lumineux arrive de l'air pour pénétrer dans du verre.
\rv
Il y a forcément un rayon réfracté
\r
Il peut y avoir réflexion totale
%\q
%Rappeler LES lois de Descartes pour la réfraction
\q
L'indice de l'air est égal à
\r 
zéro
\rv
un
\q
L'indice d'un verre est 1,43. Calculer la vitesse de la lumière dans ce verre.
\q
En quelle unité s'exprime l'indice d'un milieu ?
\q
Un rayon arrive perpendiculairement sur un dioptre. Son angle d'incidence vaut
\rv
90°
\rv
$\pi/2$
\r
0
\q
Un rayon passe de l'air (n=1) au verre (n=1,5) avec l'angle d'incidence 40°. Son angle de réfraction vaut
\rv
25,4°
\r
74,6°
\q
Lorqu'un rayon passe d'un milieu d'indice $n_1$ à un milieu d'indice $n_2>n_1$, le rayon s'écarte de la normale
\rv
Faux
\r
Vrai
\q
Lorqu'un rayon passe d'un milieu d'indice $n_1$ à un milieu d'indice $n_2<n_1$, il peut y avoir réflexion totale.
\rv
Vrai
\r
Faux
\q 
Calculer l'angle limite au delà duquel on a réflexion totale à la sortie du verre dans l'air ($n\e{verre}=1,5$)
\rv
$i_\ell=42$°
\r
pas d'angle limite
\r
$i_\ell=38°$
\q 
Utilisation de la machine : calculer
\rv
$\sin^{-1} (0,675)=42,4$°
\r 
$\sin^{-1} (0,675)=0,74$°
\r 
$\sin^{-1} (0,675)=4,48$°
\q 
%Convertir : 1,54 radians =
%\rv
%88,3°
%\r 
%0,0268°
%\r 
%79,4°
%\r 
%0,456°
%\q 
Utilisation de la machine : sin(1,67 rad)=
\rv
0,995
\r 
0,029
\r 
0,177

% Descartes : rajouter phy_2_descartesx.jpg
%\q
%Un rayon passe de l'air (indice 1,0) dans le verre (indice 1,3). Quel schéma %respecte la loi de Descartes pour cette situation ?
                  

 
%\q
%Un rayon de lumière passe du verre (indice 1,3) dans l'air (indice 1,0). %Quel schéma respecte la loi de Descartes pour cette situation ?
                  

%\q 
% Un rayon de lumière passe d'un milieu n°1 à un milieu n°2 suivant le schéma %ci-contre. Que peut-on en déduire ?
%\rv
% L'indice du milieu n°1 est plus grand que l'indice du milieu n°2.
% \r
% L'indice du milieu n°1 est égale à l'indice du milieu n°2.
% \r
% L'indice du milieu n°1 est plus petit que l'indice du milieu n°2.

	\q
  Le rayon d'un faisceau d'une lumière monochromatique est dirigé sur une lame de verre d'indice de réfraction 1,47. L'angle d'incidence est de 40° et le milieu d'incidence est l'air. L'angle de réfraction vaut:
	\r
         17°
	\r
         40°
	\rv
         26°
	\r
         51°
	\q
  L'\oe il humain est sensible aux lumières dont les longueurs d'ondes sont comprises entre:
	       \rv
         0,4 et 0,8 $µ$m
	 \r
         0,4 et 0,8 pm
	 \rv
         400 et 800 nm
	 \r
         400 et 800 $µ$m
\q
Un laser émet une lumière rouge. Sa longueur d'onde peut être de 
\rv
730 nm
\r
420 nm
