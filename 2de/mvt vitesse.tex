 \q
        Pour d�crire le mouvement d'un objet, il faut toujours d�finir 
  \r
       la vitesse
  \r
       son poids
  \rv
       un r�f�rentiel
  \r
       l'acc�l�ration
  \q
  Le r�f�rentiel li� � la surface de la Terre est le r�f�rentiel 
  \rv
       terrestre
  \r
       h�liocentrique
  \r
       g�ocentrique
  \r
       saturnocentrique
  \q
  Le r�f�rentiel li� au centre de la Terre est le r�f�rentiel 
  \r
       terrestre
  \r
       saturnocentrique
  \rv
       g�ocentrique
  \r
       h�liocentrique
  \q
  Le r�f�rentiel li� au centre du Soleil est le r�f�rentiel 
  \r
       saturnocentrique
  \rv
       h�liocentrique
  \r
       terrestre
  \r
       g�ocentrique
  \q
  Je peux marcher, tout en restant au m�me endroit dans un r�f�rentiel terrestre
  \r 
  dans un ascenseur en chute libre
  \rv
           sur un tapis roulant
   \r
       sur un circuit automobile
  \q
  La relation permettant de calculer la vitesse v pour un objet parcourant une distance d pendant une dur�e t est:
  \r
       v = t / d
  \r
       d = v / t
  \r
       v = d�t
  \rv
       v = d / t
  \q
  Un train parcourt 50 km en 36 minutes. Calculer sa vitesse moyenne.
       \rv
       $v = 23 \mps$
       \r
       $v = 0,023 \mps$
       \r
       $v = 1,4 km/h$
       \r
       $v = 1,4 \mps$
   \q
   Une voiture roule � la vitesse moyenne de 30 m/s pendant 1 heure. Calculer la distance parcourue.
       \r
       d = 0,5 km
       \r
       d = 30 km
       \r
       d = 108 m
       \rv 
       d = 108 km
   \q
   Une personne marche � la vitesse de 1,2 m.s$^{-1}$. Il doit parcourir une distance de 12,5 km. Quelle va �tre sa dur�e de parcours ? 
       \rv
       10417 s
       \rv
       environ 2h 53 min
       \r
       15000 s
       \r
       environ 4h 10 min
   \q
   Convertir 36 km.h$^{-1}$ en m.s$^{-1}$.
 \q
   Convertir 45 m.s$^{-1}$ en km.h$^{-1}$.
