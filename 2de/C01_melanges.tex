\q
Un mélange peut-être constitué 
\rv 
de deux espèces chimiques 
\rv 
de plus de deux espèces chimiques 
\rv 
de deux liquides miscibles
\q 
la masse volumique du fer et de 7,8 grammes par centimètre cube. La masse d'un objet en fer de 10 $cm^{3}$ est égale à 
\r 
7,8 g 
\rv 
78 g 
\r 
0,78 g 
\q
Un acier a une composition massique de 98 \% en fer et 2 \% en carbone. 
\r 
Une pièce en acier 200 kg contient 98 kg de fer et 2 kg de carbone 
\rv 
Une pièce en acier de 1000 kg contient 980 kg et 20 kg de carbone 
\rv 
Une pièce en acier de 500 kg contient 490 kg de fer et 10 kg de carbone
\q 
Sur la ligne de dépôt d'une plaque de CCM on dépose une substance. Après élution on observe trois taches.  La substance étudiée est
\r
un corps pur 
\r
un mélange constitué de deux espèces au plus 
\rv
un mélange constitué au moins de trois espèces
\q
l'alcool à 70 degrés, constitué d'eau de d'éthanol, est
\rv
un mélange
\r
un corps pur
\r
une espèce chimique
\q
La masse volumique a pour expression
\rv
$\rho=\f{m}{V}$
\r
$\rho=\f{V}{m}$
\r
$\rho=V\times m$
\q
La masse volumique de l'eau est 
\rv
$1000{ kg.m^{-3}}$
\rv
$1{ g.cm^{-3}}$
\r
$1{ g.m^{-3}}$
\r
$1{ g.L^{-1}}$
\q
L'unité de la densité est 
\r
le$ { kg.L^{-1}} $
\rv
sans unité
\r
le$ { g.L^{-1}} $

\q
Dans une chromatographie, l'éluant permet de 
\rv
séparer les espèces du mélange
\r
révéler le chromatogramme
\r
diluer la solution
\q
Donner la définition du rapport frontal. Faire un dessin.
\q
Le rapport frontal 
\rv
est toujours compris entre 0 et 1
\r
peut être négatif
\r
peut être supérieur à 1
\q
Le rapport frontal
\rv
n'a pas d'unité
\r
s'exprime en mètre
\r
s'exprime en centimètre
\q
Une masse volumique s'exprime en ${g.cm^{-2}}$.
\r
Vrai
\rv
Faux
\q
La densité a pour définition $d=\dfrac{\rho\e{échantillon}}{\rho\e{eau}}$
\rv
Vrai
\r
Faux
\q
Une espèce chimique a pour densité 1,2. Donner sa masse volumique
\rv
$\rho = 1,2 g.cm^{-3}$
\r 
$\rho = 1,2 kg.cm^{-3}$
\rv 
$\rho = 1,2 kg.L^{-1}$

