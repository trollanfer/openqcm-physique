     \q
       Pour r�aliser le spectre d'une lumi�re, on utilise:
     \rv
       un prisme
       \rv
       un r�seau
       \r
       un fil
       \r
       une lentille
\q
Le spectre de la lumi�re �mise par une lampe � incandescence est un spectre
\r
       de raies d'�mission
\r
       de bandes d'absorption
\r
       de raies d'absorption
\rv
       d'�mission continu
\q
Le spectre de la lumi�re �mise par une lampe � vapeur � vapeur de sodium est un spectre 
\r
       de raies d'absorption
\rv
       de raies d'�mission
\r
       d'�mission continu
\rv
       de bandes d'absorption
\q
Le spectre de la lumi�re blanche apr�s avoir travers�e un gaz monoatomique est un spectre
\rv
       de bandes d'absorption
\r
       d'�mission continu
\r
       de raies d'absorption
\r
       de raies d'�mission
\q
Le spectre de la lumi�re blanche apr�s avoir travers�e une solution color�e est un spectre
\r
       de raies d'�mission
\r
       de raies d'absorption
\rv
       de bandes d'absorption
\r
       d'�mission continu
\q
Une lampe � incandescence brille fortement. On diminue progressivement la tension d'alimentation. Comment �volue le spectre de la lumi�re �mise 
\rv
       il s'appauvrit de radiations color�es de longueurs d'ondes courtes.
\r
       il s'appauvrit de radiations color�es dans le domaine du rouge.
\rv
       il s'enrichit de radiations color�es dans le domaine du rouge.
\r
       il s'enrichit de radiations color�es de longueurs d'ondes courtes.
\r
Le spectre d'une �toile est un spectre
\r
       de bandes d'absorption
\rv
       de raies d'absorption
\r
       d'�mission continu
\r
       de raies d'�mission
\q
Le spectre d'une �toile permet
\r
       de d�terminer sa taille
\r
       de d�terminer la composition de son centre
\r
       de d�terminer le nombre de plan�tes qui gravitent.
\rv
       de d�terminer la composition de sa chromosph�re
