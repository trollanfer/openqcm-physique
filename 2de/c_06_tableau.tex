\q
Le sodium Na forme facilement
\rv
Na$^{+}$
\r
Na$^{-}$
\r
Na$^{2+}$ 
\q
Les halogènes
\rv
gagnent facilement un électron
\r
perdent facilement un électron
\q
Citer les éléments de la deuxième ligne du tableau périodique
\q
Citer les éléments de la troisième ligne du tableau périodique
\q
$(K)^2(L)^8(M)^2 $ correspond à l'atome de l'élément
\rv
magnésium Mg
\r
sodium Na
\r
berylium Be
\q
Les éléments de la dernière colonne du tableau périodique sont appelés
\r
les halogènes
\r
les alcalins
\rv
les gaz nobles
\q
Les éléments de l'avant dernière colonne du tableau périodique sont appelés
\rv
les halogènes
\r
les alcalins
\r
les gaz nobles
\q
Les éléments de la première colonne sont
\r
les halogènes
\rv
les alcalins
\r 
les gaz nobles

\q
Dans la classification périodique, les éléments sont rangés par
\rv
numéro atomique croissant
\r
nombre de masse croissant
\rv
nombre de protons croissant
\q
Les atomes de formule $K^2L^5$ et $K^2L^8M^5$ appartiennent à la même
\r
période
\r
ligne
\rv 
colonne
\q
Les atomes des éléments d'une même famille ont tous
\r
le même nombre d'électrons
\rv
le même nombre d'électrons externes
\rv
des propriétés chimiques voisines
\q
Un atome qui donne un ion $M^{2+}$ peut appartenir à la 
\rv
deuxième colonne
\r
seizième colonne
\q
Justifier que les gaz nobles sont quasi inertes chimiquement.
\q
   Combien d'électrons possède la couche K lorsqu'elle est saturée ?
 
 \q
   Combien d'électrons possède la couche L lorsqu'elle est saturée ?
 \q
 
   Combien d'électrons possède la couche M lorsqu'elle est saturée ?
 
\q
   Pour l'atome d'argon la couche externe est ...
 
\q
Les éléments de numéro atomique proche de celui de l'hélium cherchent à acquérir .... électrons sur leur couche externe. c'est la règle du ......
Les autres éléments cherchent à acquérir .... électrons sur leur couche externe. C'est la règle de ......
 
\q
Quel est l'ion monoatomique stable formé par l'élément lithium Li (Z = 3) ?
\r
      Li$^{-}$
\rv
      Li$^{+}$
\r
      Li$^{3+}$
  \r
      Li$^{3-}$
\q
Quel est l'ion monoatomique stable formé par l'élément fluor F (Z = 9) ?
\r
      F$^{+}$
\r
      F$^{2+}$
\rv
      F$^{-}$
\r
      F$^{2-}$
\q
Citer la règle de l'octet.
\q
Quel est l'ion monoatomique stable formé par l'élément calcium Ca (Z = 20) ? 
\rv
      Ca$^{2+}$
\r
      Ca$^{+}$
\r
      Ca$^{2-}$
\r
      Ca$^{-}$
\q
Quel est l'ion monoatomique stable formé par l'élément soufre S (Z = 16) ? 
\r
      S$^{3+}$
\r
      S$^{3-}$
\rv
      S$^{2-}$
\r
      S$^{2+}$
\q
Combien de liaisons covalentes forme l'atome de carbone C (Z=6) ? 
 \rv
      4
\r
      1
\r
      2
\r
      3
\q
Combien de liaisons covalentes forme l'atome d'oxygène C (Z=8) ?
\r
      4
\r
      3
\rv
      2
\r
      1
\q
Combien de liaisons covalentes forme l'atome de phosphore P (Z=15) ?
\r
      2
\r
      1
\rv
      3
\r
      4
\q
Combien de liaisons covalentes forme l'atome de chlore Cl (Z=17) ?
\r
      3
\rv
      1
\r
      2
\r
      4
