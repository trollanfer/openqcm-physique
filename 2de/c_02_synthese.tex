\q
Dans un médicament, l'espèce chimique agissante est
\rv
un principe actif
\r
toujours une espèce naturelle
\r
toujours une espèce artificielle
\r
un excipient
\q
Une espèce chimique synthétique
\rv
est fabriquée par l'homme
\r
n'existe pas dans la nature
\r
est la copie d'une espèce naturelle
\q
Un chauffage à reflux
\r
évapore le solvant
\rv
accélère la transformation
\rv
évite la perte du solvant
\q
Une espèce chimique synthétisée
\r
n'existe pas dans la nature
\rv
peut être identique à une espèce naturelle
\q
On peut identifier un produit obtenu lors d'une synthèse par
\rv
chromatographie
\r
filtration
\r
chauffage à reflux
\q
Le support élévateur est indispensable lorsqu'on utilise un chauffe ballon. Pourquoi?
\q
Lorsque l'ébullition est trop violente, il faut d'urgence
\rv
baisser le support élévateur
\r
ajouter de l'eau froide
\r
baisser le chauffage
\q
On ne doit jamais boucher le réfrigérant d'un chauffage à reflux pour
\rv
que le montage n'explose pas
\r
évaporer le solvant
\r
permettre d'ajouter du mélange ou du solvant
\q
Donner la définition d'une synthèse.
\q
Pourquoi utilise-t-on un réfrigérant dans un chauffage à reflux?
\q
Quel est le sens du mot 'reflux' dans 'chauffage à reflux'?
\q
Donner deux raisons pour lesquelles les chimistes effectuent des synthèses.
\q
Un réfrigérant est indispensable dans un chauffage à reflux.
\rv
Vrai
\r
Faux



