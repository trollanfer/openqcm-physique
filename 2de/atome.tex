  \q
    Un atome est constitué d'électrons et d'un ....
    \rv 
    noyau
    \r
    neutron
    \r
    nucléon
    \r
    proton
  \q
    Le noyau de l'atome est constitué de ...
    \r 
    électrons
    \rv
    neutrons
    \rv
    nucléons
    \rv
    protons
      \q 
      Les nucléons sont des protons et des ...
    \q
         le proton porte une charge 
      \rv
       positive.
      \r
      négative.
      \r
      nulle
  \q 
  Le neutron porte une charge 
      \rv
      nulle.
      \r
      négative.
      \r
      positive.
  \q
  L'électron porte une charge
      \rv
      négative.
      \r
      nulle.
      \r
      positive.
  \q
  Un atome est électriquement ...
  \rv
  neutre
  \r
  chargé
  \rv
  non chargé
 \q
 Un anion est un ion ...
 \r
 de charge positive
 \rv
 de charge négative
  \q
  Un cation est un ion ...
 \rv
 de charge positive
 \r
 de charge négative
  \q
  Un anion monoatomique est un atome qui a ... un ou plusieurs électrons.
  \rv 
  gagné
  \r
  perdu
 \q
  Un cation monoatomique est un atome qui a ... un ou plusieurs électrons.
 \r 
  gagné
  \rv
  perdu
  \q
  Dans la représentation symbolique du noyau $_Z^AZ$ , que représente Z ?
   \rv
      le nombre de protons
      \r
      le nombre de neutrons
      \r
      le nom de l'élément chimique
      \r
      le nombre de nucléons
 \q
  Dans la représentation symbolique du noyau  , que représente A  ?
      \r
      le nombre de protons
      \r
      le nom de l'élément chimique
      \rv
      le nombre de nucléons
      \r
      le nombre de neutrons
 \q
  Dans la représentation symbolique du noyau  , que représente X  ?
      \rv
      le nom de l'élément chimique
      \r
      le nombre de nucléons
      \r
      le nombre de neutrons
      \r
      le nombre de protons
 \q
  D'après la représentation symbolique du noyau  , A-Z permet de calculer ...
      \rv
      le nombre de neutrons
      \r
      le nombre de nucléons
      \r
      le nom de l'élément chimique
      \r
      le nombre de protons
 \q
  Combien de protons possède l'élément aluminium $_{13}^{27}\Al$? 
 \q
  Combien de neutrons possède l'élément aluminium $_{13}^{27}\Al$? 
 \q
  Combien d'électrons possède l'atome d'aluminium $_{13}^{27}\Al$?
 \q
  Combien d'électrons possède l'ion aluminium $_{13}^{27}\Al\PPP$ ? 
 \q
  Combien d'électrons possède l'ion iodure $_{53}^{127}I\M $ ?
 \q
  Combien de protons possède l'ion iodure $_{53}^{127}I\M $  ?
 \q
  Lorsqu'on calcule la masse approchée d'un atome, on néglige la masse des
      \r
      neutrons
      \rv
      électrons
      \r
      nucléons
      \r
      protons
 \q
  Des noyaux isotopes possèdent le même nombre de protons mais un nombre de ...... différents.
  \r
  électrons
  \rv
  neutrons
  \rv
  nucléons
 \q
  Combien d'électrons possède la couche K lorsqu'elle est saturée ?
 \q
  Combien d'électrons possède la couche L lorsqu'elle est saturée  ?
 \q
  Combien d'électrons possède la couche M lorsqu'elle est saturée   ?
 \q
  La structure électronique de l'atome de béryllium (Z=3) est
      \r
      $(K)^1(L)^2$
      \rv
      $(K)^2(L)^1 $
      \r
      $(K)^3$
      \r
      $(L)^3$
 \q
  La structure électronique de l'atome de phosphore (Z=15) est
      \r
      $(K)^1(L)^8(M)^6$
      \r
      $(K)^8(L)^7    $
      \rv
      $(K)^2(L)^8(M)^5$
      \r
      $(K)^7(L)^8    $
 \q
  La structure électronique de l'ion fluorure F\M(Z=9) est
      \r
      $(K)^1(L)^8     $
      \r
      $(L)^8(M)^2      $
      \r
      $(K)^2(L)^7      $
      \rv
      $(K)^2(L)^8      $
 \q
  La structure électronique de l'ion sodium Na\P (Z=11) est
      \r
      $(L)^8(M)^2     $
      \r
      $(K)^2(L)^8(M)^1 $
      \rv
      $(K)^2(L)^8     $
      \r
      $(K)^1(L)^8(M)^2 $
 \q
  La structure électronique de l'ion potassium K\P (Z=19) est
      \r
      $(K)^2(L)^8(M)^9$
      \r
      $(K)^1(L)^8(M)^{10}$
      \r
      $(K)^1(L)^8(M)^8 $
      \rv
      $(K)^2(L)^8(M)^8 $
 \q
  La structure électronique de l'atome d'argon Ar (Z=18) est
      \r
      $(L)^8(M)^10  $
      \rv
      $(K)^2(L)^8(M)^8$
      \r
      $(K)^2(L)^{16}$
      \r
      $(K)^1(L)^8(M)^8$
