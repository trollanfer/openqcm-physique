\q	
Donner l’équation de la réaction entre l’acide chlorhydrique ($H_3O^+_{(aq)} + Cl^{-}_{(aq)}$) et une solution d’hydroxyde de sodium ($Na^{+}_{(aq)} + HO^{-}_{(aq)}$).
  \q	
On considère un volume V = 100 mL de solution d'acide chlorhydrique $(H_3O^+_{(aq)} + Cl^-_{(aq)})$, à 25 °C, de concentration c en soluté apporté (c = 1,0 . 10$^{-3}$ mol.L$^{-1}$). Le pH de cette solution est égal à 3. 
On donne Ke = 1,0 . 10$^{-14}$ à 25 °C.
\rv
La réaction du chlorure d'hydrogène $HCl_{(aq)}$ avec l'eau est une transformation chimique totale ;
\r
L'avancement maximal de la réaction du chlorure d’hydrogène $HCl_{(aq)}$ avec l'eau est $x_m$ = 1,0 . 10$^{-3}$ mol;
\rv
Dans cette solution, $[HO^{-}]_f$ = 1,0 . 10$^{-11}$ mol.L$^{-1}$   
 \q	
On considère une solution aqueuse contenant de l’acide éthanoïque.
Le pKa du couple $CH_3COOH_{(aq)}/ CH_3COO^{-}_{(aq)}$  vaut 4,8 ;
\rv
La forme $CH_3COO^{-}_{(aq)}$  prédomine  pour une solution de pH = 6,5 ;
\r
Les formes $CH_3COOH_{(aq)}$ et $CH_3COO^{-}_{(aq)}$ sont présentes en quantités égales pour une solution de pH=7 ;
\r
Si le pH de la solution vaut 8, celle-ci ne contient pas d’ions $H_3O^+_{(aq)}$.
\q
On dispose d’un verre de jus d’orange (pH = 3,0) et d’une tasse de café (pH = 6,0). 
\rv
Le jus d’orange contient des ions $HO^{-}_{(aq)}$ ; 
\r
La concentration en ions $H_3O^+_{(aq)}$  est plus importante dans le café ;
\rv
Les concentrations en ions $H_3O^+_{(aq)}$ sont dans un rapport de  1000.
\q
Rappeler l’expression et la valeur du produit ionique de l’eau.
 \q	
A 25° C, la concentration molaire des ions oxonium d’une solution est 1,0.10$^{-3}$ mol.L$^{-1}$ . 
On donne Ke = 1,0 . 10$^{-14}$ 
\rv
pH = 3,0 ;
\rv La concentration molaire des ions hydroxyde est 1,0.10$^{-11}$ mol.L$^{-1}$ ;
\r
La solution est basique.

\q	
On donne le produit ionique de l’eau : Ke = 1,0.10$^{-14}$ à 25°C.
\rv
Pour un couple HA/A-, l’espèce HA est prédominante lorsque pH est inférieur au pKa du couple ;
\rv
Plus la constante d’acidité Ka d’un couple acide/base est grande, plus l’acide est dissocié dans l’eau ;
\r
On peut écrire [$H_3O^+]_f = [HO^{-}]_f$ = 1,0.10$^{-7}$ mol.L$^{-1}$ quelle que soit la solution aqueuse étudiée ;
\r	
Le pH d’une solution aqueuse d’hydroxyde de sodium dans laquelle la concentration molaire en ions hydroxyde est 1,0.10$^{-3}$ mol.L$^{-1}$ a pour valeur pH = 3 
\q	
La réaction d’équation $2 H_2O(l) = H_3O^+_{(aq)}  + HO^{-}_{(aq)}$ présente la ou les propriétés suivantes :
\r
Elle ne se produit que dans l’eau pure ; 
\r
$[H_3O^+] = [HO^{-}]$  dans toutes les solutions où elle se produit ;
\rv
Elle a un avancement très faible dans l’eau pure.
 \q	
Soit la réaction d’équation :	$2 H_2O(l) = H_3O^+_{(aq)}  +  HO^{-}_{(aq)}.$
\rv
Elle correspond à une réaction acido-basique ;
\rv
La constante d’équilibre associée est appelée Ke ; 
\rv
Elle est appelée réaction d’autoprotolyse de l’eau ;
\r
Elle n’a lieu que dans l’eau pure.
   \q	On considère la réaction :	 $AH_{(aq)} + H_2O(l) = A^-_{(aq)} + H_3O^+_{(aq)}$
\rv
Il s’agit d’une réaction acido-basique ;
\r 	
A l’équilibre : pH = pKa + log $\dfrac{[AH]}{[A^-]}$  ;
\r
Si pH < pKa, l’espèce $A^-_{(aq)}$ est prédominante.
\q
On dissout 0,10 mol d’ammoniac $NH_3(g)$ dans de l’eau distillée de façon à obtenir 200 mL de solution.
Le pH de la solution est de 11,4 :
On donne :    Ke=10$^{-14}$ ;     pKa ($NH_4^+/NH_3$)=  9,2 .
\rv
La réaction de l’ammoniac sur l’eau est modélisée par $NH_{3(aq)} + H_2O(l) = NH_4^+ + HO^{-}_{(aq)}$ ;
\r
L’ammoniac est une espèce acide ;
\q	
Un acide AH réagit totalement avec l’eau. Une solution aqueuse de cet acide a une concentration égale à 1,0 . 10$^{-3}$ mol.L$^{-1}$ 
\rv
Le pH de cette solution est 3,0 ;
\r
Lorsqu’on dilue cette solution, le pH diminue ;
\r
Cette solution étant acide elle ne contient pas d’ions $HO^{-}$.

  
