\q
Quel(s) type(s) de mouvement peut-on avoir lorsque le vecteur vitesse est constant ?
\q	
Énoncer la 1ère loi de Newton (ou principe d'inertie).
\q
	Donner deux exemples de référentiels supposés galiléens.
 \q	
 Indiquer l’expression vectorielle de la deuxième loi de Newton.
\q
	Le centre d’inertie G d’un système est animé d’un mouvement circulaire uniforme.
\r
  Le vecteur accélération de G est constant ;
\rv
  La valeur de la vitesse est constante ;
\rv
  La valeur de l’accélération est constante ;
\rv
  Le vecteur accélération   est normal à la trajectoire.
\q	
Donner l’expression vectorielle de la troisième loi de Newton (loi des actions réciproques).
\q
Lorsque l'on s'appuie sur un mur, celui-ci exerce une force
\rv
égale à celle que l'on exerce sur lui
\r
nulle
\rv
juste nécessaire à empêcher la main de s'enfoncer dans le mur
\q
Une force peut s'exprimer en
\rv
N
\rv
kg.m.s$^{-2}$
\r
m.s$^{-2}$
\r
kg.m.s$^{-1}$
\q
Un mobile se déplace suivant l'axe des x selon l'équation horaire $x(t)=-3t^{2}+2t+6$. Son accélération vaut :
\rv
autre réponse
\r
$-6t+2$
\r
-12
\q
Un mobile se déplace suivant l'axe des x selon l'équation horaire $x(t)=-3t^{2}+2t+6$. Sa vitesse s'écrit :
\rv
$-6t+2$
\r
-6
\r
$-3t+2$
\q
Un mobile se déplace suivant l'axe des x selon l'équation horaire $x(t)=-3\sin(2t)+4t$. Sa vitesse s'écrit 
\rv
$-6\cos(2t)+4$
\r
autre réponse
\r
$6\cos(2t)+4$
\r
$-3\cos(2t)+4$
\r
$3\cos(2t)+4$
\q
Un mobile se déplace sur l'axe des x. A l'instant $t=0$, il se trouve à l'abscisse x = 2 m, avec une vitesse nulle. Son accélération est de 3m.s$^{-2}$.
L'équation horaire du mouvement est :
\rv
$1,5t^{2}+2$
\r
$3t^{2}+2$
\r
$1,5t^{2}+2t$
\r
$3t^{2}+2t$
\r
autre réponse
