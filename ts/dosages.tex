\q	
  On considère la réaction de titrage de l’acide éthanoïque par la soude : 
$CH_3COOH_{(aq)}  +  HO^{-} _{(aq)} =  CH_3COO^{-}_{(aq)}  +  H_2O$ 
Données : 	 pKa1($CH_3COOH / CH_3COO^{-}$ ) = 4,8 ;
			pKa2($H_2O /HO^{-} $) = 14 
  \rv
  La constante d’équilibre associée à la réaction de titrage est égale à 10$^{9,2}$ ;
  \rv
  Elle peut être suivie par pH-métrie ;
  \r
  A l’équivalence, le mélange est neutre.
 \q	On réalise le titrage d’une solution d’acide éthanoïque par une solution d’hydroxyde de sodium. Le pH à l’équivalence est voisin de 4.
 \rv
  La réaction de titrage est considérée comme totale ;
\rv 
  La solution d’acide éthanoïque est versée progressivement à l’aide d’une burette ;
\rv
  L’hélianthine est l’indicateur coloré adapté à ce titrage (zone de virage : 3,1 - 4,4). 
\q
	On réalise le titrage d’une solution d’acide éthanoïque par une solution d’hydroxyde de sodium.
\rv
L’équation de la réaction est : $CH_3COOH_{(aq)} +  HO^{-}_{(aq)}  =  CH_3COO^{-}_{(aq)}  +  H_2O(l)$ .
\rv
  Les réactifs dans le mélange sont totalement consommés à l’équivalence ;
\rv
  La transformation peut être considérée comme totale ;
\r
L’équation de la réaction est : $H_3O^+ +  HO^{-}  = 2 H_2O$
    \q	
    Donner les caractéristiques d’une réaction de titrage.
\q	
On considère le titrage acido-basique entre une solution d’acide $CH_3COOH$ (de concentration $C_A$ et de volume $V_A$) et une solution basique d’hydroxyde de sodium (de concentration $C_B$). 
A l’équivalence, on appelle $V_{BE}$ le volume versé et : 
\rv
  $C_A.V_A = C_B.V_{BE}$ ;
\r
  La solution obtenue est neutre (pHe =7)  ;
\r
  finalement, $n_f(CH_3COOH) = 0$ et $n_f(HO^{-}) = 0$ ;
 \r
  L’avancement maximal est nul.
\q
	Définir l’équivalence d’un titrage.
  \q	La réaction  $2 H_2O (l) = H_3O^+_{(aq)} + HO^{-}$  
\rv
  Est une réaction acido-basique ;
\r
  Correspond à une réaction de titrage acido-basique ;
\rv
  Est appelée autoprotolyse de l’eau.
  \q	On réalise le titrage pH-métrique de 20,0 mL d’une solution aqueuse d’acide méthanoïque de concentration inconnue par une solution d’hydroxyde de sodium de concentration molaire Cb = 1,0.10$^{-2}$  mol.L$^{-1}$ . 
Le volume versé à l’équivalence est Veq = 18,0 mL.
 \rv
  La réaction de titrage doit être totale ;
\r
  Le prélèvement d’acide est réalisé à l’aide d’une éprouvette graduée ;
\r
  La réaction de titrage est 		$CH_3COOH + HO^{-} = CH_3COO^{-}  +  H_2O$ ;
\rv
  La concentration de la solution d’acide est C = 9,0.10$^{-3}$ mol.L$^{-1}$ .
 \q
 	Lors d’un titrage acido-basique:
\r
  Le volume de la prise d’essai est mesuré avec une éprouvette graduée ;
\rv
  La concentration du réactif titrant est connue ;
\r
  L’utilisation d’un pH-mètre est indispensable pour repérer l’équivalence.
 \q	
 Comment déterminer graphiquement le point d’équivalence lors d'un titrage acido-basique ?
 \q	
 L'équation de la réaction de titrage entre les ions hydroxyde et oxonium s’écrit : 
$H_3O^+_{(aq)} + HO^{-}_{(aq)} = 2 H_2O(l)$ , avec $Na^{+}_{(aq)}$ et $Cl^{-}_{(aq)}$ comme ions spectateurs.
 \rv
  Sa constante d’équilibre s’écrit K = $[H_3O^+]_{eq}.[HO^{-}]_{eq} $;
\r
  A l’équivalence, on obtient toujours de l’eau salée ;
\rv
  C’est une transformation totale.
