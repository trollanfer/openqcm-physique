 \q	
 On considère une solution aqueuse contenant de l’acide éthanoïque.
Le pKa du couple $CH_3COOH_{(aq)}/ CH_3COO^{-}_{(aq)}$  vaut 4,8 ;
\rv
  La forme $CH_3COO^{-}_{(aq)}$  prédomine  pour une solution de pH = 6,5 ;
\r
  Les formes $CH_3COOH_{(aq)}$ et $CH_3COO^{-}_{(aq)}$ sont présentes en quantités égales pour une solution de pH=7 ;
\r
  Si le pH de la solution vaut 8, celle-ci ne contient pas d’ions $H_3O^+_{(aq)}$.
  \q
  	On dispose d’un verre de jus d’orange (pH = 3,0) et d’une tasse de café (pH = 6,0). 
\rv
 	Le jus d’orange contient des ions $HO^{-}_{(aq)}$ ; 
\r
 	La concentration en ions $H_3O^+_{(aq)}$  est plus importante dans le café ;
\rv
 	Les concentrations en ions $H_3O^+_{(aq)}$ sont dans un rapport de  1000.
 \q	
 On dispose d'une solution A d'acide éthanoïque $CH_3COOH$ de concentration  $c_A$  = 1,0 . 10$^{-2}$ mol.L$^{-1}$ dont le pH est égal à 3,4 et d'une solution B d'éthanoate de sodium $Na^{+}_{(aq)} + CH_3COO^{-}_{(aq)}$ de concentration  $c_B$  = 1,0 . 10$^{-2}$ mol.L$^{-1}$ dont le pH est égal à 8,2. 
$pK_A$($CH_3COOH/ CH_3COO^{-}$) = 4,75 
On verse 100 mL de chacune des solutions dans un bécher.
\r
  l'acide éthanoïque réagit de façon totale avec l'eau ;
\rv
  le pH du mélange est égal au pKA ;
\r
  le pH du mélange est modifié si l'on rajoute 100 mL d'eau.
\q
	Rappeler l’expression du produit ionique de l’eau.
 \q	
 Exprimer la constante d’acidité Ka du couple $NH_4^+/ NH_3$.
 \q	
 A 25° C, la concentration molaire des ions oxonium d’une solution est 1,0.10$^{-3}$ mol.L$^{-1}$ . 
On donne Ke = 1,0 . 10$^{-14}$ 
  \rv
  pH = 3,0 ;
 \rv La concentration molaire des ions hydroxyde est 1,0.10$^{-11}$ mol.L$^{-1}$ ;
\r
  La solution est basique.

\q	
On donne le produit ionique de l’eau : Ke = 1,0.10$^{-14}$ à 25°C.
\rv
	  Pour un couple HA/A-, l’espèce HA est prédominante lorsque pH est inférieur au pKa du couple ;
\rv
	  Plus la constante d’acidité Ka d’un couple acide/base est grande, plus l’acide est dissocié dans l’eau ;
\r
	  On peut écrire [$H_3O^+]_f = [HO^{-}]_f$ = 1,0.10$^{-7}$ mol.L$^{-1}$ quelle que soit la solution aqueuse étudiée ;
\r	
	  Le pH d’une solution aqueuse d’hydroxyde de sodium dans laquelle la concentration molaire en ions hydroxyde est 1,0.10$^{-3}$ mol.L$^{-1}$ a pour valeur pH = 3 
 \q	
 La réaction d’équation $2 H_2O(l) = H_3O^+_{(aq)}  + HO^{-}_{(aq)}$ présente la ou les propriétés suivantes :
 \r
  Elle ne se produit que dans l’eau pure ; 
\r
  $[H_3O^+] = [HO^{-}]$  dans toutes les solutions où elle se produit ;
\rv
  Elle a un avancement très faible dans l’eau pure.
 \q	
 Soit la réaction d’équation :	$2 H_2O(l) = H_3O^+_{(aq)}  +  HO^{-}_{(aq)}.$
\rv
  Elle correspond à une réaction acido-basique ;
\rv
  La constante d’équilibre associée est appelée Ke ; 
\rv
  Elle est appelée réaction d’autoprotolyse de l’eau ;
 \r
   Elle n’a lieu que dans l’eau pure.
  \q
 	Quelle est l’expression de la constante d’acidité d’un acide noté AH ?
  \q	Exprimer la constante d’acidité Ka du couple $HCOOH/HCOO^{-}$.
   \q	On considère la réaction :	 $AH_{(aq)} + H_2O(l) = A^-_{(aq)} + H_3O^+_{(aq)}$
\rv
 	Il s’agit d’une réaction acido-basique ;
\r 	
A l’équilibre : pH = pKa + log $\dfrac{[AH]}{[A^-]}$  ;
\r
 	Si pH < pKa, l’espèce $A^-_{(aq)}$ est prédominante.
 %\q	
 %On étudie la réaction de l’acide méthanoïque dans l’eau :
%$HCOOH_{(aq)} + H_2O(l) = HCOO^{-} _{(aq)} + H_3O^+_{(aq)}$ 
%\rv
%  La conductivité  de la solution est égale à : 
%		$\sigma =  \lambda(H_3O^+)×[ H_3O^+] -  \lambda(HCOO^{-})×[ HCOO^{-}]$ ;
%\r
%  Le quotient de réaction s’écrit : $Qr =[ H_3O^+]×[ HCOO^{-}]$ ;
%\rv
%  Le quotient de réaction à l’équilibre Qr,f  est égal à la constante d’acidité du couple Ka .
 \q	
 Donner l’expression du Ka du couple dont l’acide a pour formule $HClO_{(aq)}$.
  \q
  	On dissout 0,10 mol d’ammoniac $NH_3(g)$ dans de l’eau distillée de façon à obtenir 200 mL de solution.
Le pH de la solution est de 11,4 :
On donne :    Ke=10$^{-14}$ ;     pKa ($NH_4^+/NH_3$)=  9,2 .
\rv
  La réaction de l’ammoniac sur l’eau est modélisée par $NH_{3(aq)} + H_2O(l) = NH_4^+ + HO^{-}_{(aq)}$ ;
\r
  L’ammoniac est une espèce acide ;
\r
  La constante d’équilibre K de la transformation est supérieure à  Ka
 \q	
 Un acide AH réagit totalement avec l’eau. Une solution aqueuse de cet acide a une concentration égale à 1,0 . 10$^{-3}$ mol.L$^{-1}$ 
\rv
  Le pH de cette solution est 3,0 ;
\r
  Lorsqu’on dilue cette solution, le pH diminue ;
\r
  Cette solution étant acide elle ne contient pas d’ions $HO^{-}$.
  \q	
  Soit la réaction suivante :$C_6H_5COOH_{(aq)} + H_2O  = C_6H_5COO^{-}_{(aq)} + H_3O^{+}$.   
%\r
%  On définit la conductivité du mélange par :  $\sigma$  = $\lambda(C_6H_5COOH) .[C_6H_5COOH] + \lambda(C_6H_5COO^{-}).[C6H5COO^{-}]$ ;
\rv
  C’est une réaction acido-basique ;
\r
  La constante d’équilibre s’écrit : $Q_{r’éq} =\dfrac{[C_6H_5COO^{-}][H_3O^{+}]}{[C_6H_5COOH][H_2O]}$   .
   \q	
   On prépare une solution $S_2$ d’acide formique (ou acide méthanoïque) de concentration        $c_2 = 5,0 . 10^{-3}$ mol.L$^{-1}$ à partir d’une solution $S_1$ d’acide formique de concentration $c_1 =1,0 . 10^{-1}$ mol.L$^{-1}$ 
\rv
  Pour la dilution on doit utiliser une éprouvette de 5 mL et une fiole jaugée de 100 mL ;
\rv 
 La réaction de l’acide avec l’eau s’écrit :
$HCOOH_{(aq)}+H_2O(l) = HCOO^{-} _{(aq)}+ H_3O^+ _{(aq)}$ ;
\rv
  La constante d’équilibre de la réaction de l’acide formique dans l’eau s’écrit :
		$Ka=\dfrac{[HCOO^{-}][H_3O^+]}{[HCOOH]}$   .
 \q	
 On s’intéresse aux espèces chimiques appartenant aux deux couples suivants :
$HA_1/A_1^-$ (pKa1 = 3 ) et $HA_2/A_2^-$ ( pKa2 = 8 ).
\r
  Une réaction entre $A_1^-$ et $A_2^-$ est possible ;
\rv 
  La constante d’équilibre de la réaction entre $HA_1$ et $A_2^-$ a pour valeur 10$^{5}$ ;
\rv
  A la même concentration, l’acide le plus dissocié dans l’eau est $HA_1$ 
