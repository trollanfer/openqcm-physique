\q
L'ion ascorbate $C_6H_7O_6^{-}$ constitue la forme basique d'un couple acide /base . 
Quel est ce couple ? 
\q
Donner la définition d'une base.
\q
Compléter l'équation de la réaction acido-basique suivante :
$HCO_2H_{(aq)}+ HO^{-}_{(aq)} =       …..     +  …….$
\q	
On désire préparer une solution A de concentration molaire 1,0.10$^{-1}$ mol.L$^{-1}$ à partir  d'une solution mère de concentration molaire 1,0 mol.L$^{-1}$  Pour cela on peut utiliser :
\r
Une fiole jaugée de 100 mL et une pipette jaugée de 1 mL ;
\r
Un bécher de 100 mL et une pipette jaugée de 10 mL ;
\rv
Une fiole jaugée de 50 mL et une pipette jaugée de 5 mL ;
\rv
Une fiole jaugée de 100 mL et une pipette jaugée de 10 mL.
\q	
 Donner la définition du pH
\q	
  Qu'est-ce qu'une transformation chimique limitée/équilibrée ?
\q	
  Quels sont les couples acide/base dans lesquels on retrouve la molécule d'eau   ?
\q	
 On désire fabriquer 100 mL d'une solution de chlorure de potassium ($K+_{(aq)} + Cl^{-}_{(aq)}$) de concentration   c = 0,020 mol.L$^{-1}$ en diluant 5 fois une solution mère. Choisir la verrerie nécessaire :
\r
  becher de 100 mL ;		
\rv
    pipette jaugée de 20 mL ;
\r
  erlenmeyer de 100 mL ;	
\r
  	  burette graduée de 50 mL ;
\rv
  fiole jaugée de 100 mL ; 
\r
              	  éprouvette graduée de 20 mL.
\q
   	Le pH d'une solution d'acide nitreux $HNO_2$ de concentration c = 1,00.10$^{-2}$ mol.L$^{-1}$ vaut 2,7 :
\r  
L'équation de la réaction de l'acide nitreux avec l'eau s'écrit : 
$HNO_2 + HO^{-}_{(aq)}  = NO_3^{-} + H_2O_(l)$ ;
\rv
  L'acide nitreux appartient au couple acide base $HNO_2/NO_2^-$ ;
\rv
  L'acide nitreux réagit de façon limitée avec l'eau.
\q	
   Parmi les couples suivants, donner le (ou les) couple(s) acide/base ?
\r
 	$MnO_4^- / Mn^{2+}_{(aq)}$ ;
\rv
 	$CH_3COOH_{(aq)} / CH_3COO^{-}_{(aq)}$ ;
\rv
 	$H_2O_{(aq)} / HO^{-}_{(aq)}$ ;
\r
 	$S_2O_8^{2-}/ SO_4^{2-}$ ;
\rv
 	$CH_3NH_3^+ / CH_3NH_2$.

\q	
   On considère un volume V = 100 mL de solution d'acide chlorhydrique $(H_3O^+_{(aq)} + Cl^-_{(aq)})$, à 25 °C, de concentration c en soluté apporté (c = 1,0 . 10$^{-3}$ mol.L$^{-1}$). Le pH de cette solution est égal à 3. 
\rv
  La réaction du chlorure d'hydrogène $HCl_{(aq)}$ avec l'eau est une transformation chimique totale ;
\r
  L'avancement maximal de la réaction du chlorure d'hydrogène $HCl_{(aq)}$ avec l'eau est xmax = 1,0 . 10$^{-3}$ mol;
  \rv
  L'avancement maximal de la réaction du chlorure d'hydrogène $HCl_{(aq)}$ avec l'eau est xmax = 1,0 . 10$^{-4}$ mol;
 \q	
Soient trois solutions $S_1$, $S_2$ et $S_3$ dont on connaît les propriétés suivantes à 25° C:
Pour $S_1$, $[H_3O^+]$ = 4,0 .10$^{-5}$ mol.L$^{-1}$ ;
Pour $S_2$, pH = 5 ;
Pour $S_3$, 100 mL contiennent 6,0.10$^{-6}$ mol d'ions $H_3O^+$.
A partir de la détermination de $[H_3O^+]$, indiquer si :
\rv
Toutes les solutions sont acides ;
\r
Les trois solutions ont le même pH ;
\r
$S_2$ est la plus acide.
\q	
On dispose d'une solution d'un acide HA de concentration en soluté apporté $c_a$ = 1,0.10$^{-2}$  mol.L$^{-1}$. Son pH est égal à  2,5.
\rv
La réaction de l'acide avec l'eau est limitée 
\r
La réaction de l'acide avec l'eau est totale
\q	
Exprimer la constante d'acidité Ka du couple $NH_4^+/ NH_3$.
\q	Exprimer la constante d'acidité Ka du couple $HCOOH/HCOO^{-}$.
   \q	
On prépare une solution $S_2$ d'acide formique (ou acide méthanoïque) de concentration        $c_2 = 5,0 . 10^{-3}$ mol.L$^{-1}$ à partir d'une solution $S_1$ d'acide formique de concentration $c_1 =1,0 . 10^{-1}$ mol.L$^{-1}$ 
\r
Pour la dilution on doit utiliser une éprouvette de 5 mL et une fiole jaugée de 100 mL ;
\rv 
La réaction de l'acide avec l'eau s'écrit :
$HCOOH_{(aq)}+H_2O(l) = HCOO^{-} _{(aq)}+ H_3O^+ _{(aq)}$ ;
\rv
La constante d'équilibre de la réaction de l'acide formique dans l'eau s'écrit :
$$Ka=\dfrac{[HCOO^{-}][H_3O^{+}]}{[HCOOH]}$$
